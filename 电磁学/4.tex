\documentclass{phyasgn}
\phyasgn{
  stuname = 姚昊廷,           % 设置学生姓名
  stunum = 22322091,      % 设置学号
  setasgnnum = 4,           % 设置课程次数
  classname = 电磁学,     % 设置课程名称
}

\usepackage{listings}
\usepackage{tikz}
\usepackage{amssymb}
\usepackage{t-angles}
\usepackage{amssymb}
\usepackage{tikz} 
%\usepackage{fixdif} 
\usetikzlibrary{quotes,angles}
\usetikzlibrary{calc}
\usetikzlibrary{decorations.pathreplacing}
\lstset{numbers=left,basicstyle=\ttfamily,columns=flexible}
\makeatletter
\newcommand{\rmnum}[1]{\romannumeral #1}
\newcommand{\Rmnum}[1]{\expandafter\@slowromancap\romannumeral #1@}
\makeatother


\begin{document}
{\zihao{5}\heiti\color{red} 1-45}
\begin{sol}
(1)$$\vec{E}_A=\frac{\sigma_e}{2\varepsilon_0}\hat{x}$$
(2)$$\vec{E}_B=\frac{\sigma_e}{2\varepsilon_0}\hat{x}$$
(3)$$\vec{E}=\vec{E}_A+\vec{E}_B=\frac{\sigma_e}{\varepsilon_0}\hat{x}$$
(4)均匀分布在平板左右两侧
$$E_A^{\prime}=\frac{\sigma_e}{2\varepsilon_0}\hat{x}$$
\end{sol}\par

{\zihao{5}\heiti\color{red} 1-46}
\begin{sol}
\begin{figure}[!h]
    \begin{tikzpicture}
        \draw (-2,5) --(-2,-5) node[below]{1};
        \draw (-1,5) --(-1,-5) node[below]{2};
        \draw (1,5) --(1,-5) node[below]{3};
        \draw (3,5) --(3,-5) node[below]{4};
        \draw (-1.5,4) rectangle (2,3);
        \draw (4,-2) rectangle (-3,-1);
        \draw (-1.5,-4) rectangle (-3,-3);
        \draw (2,-4) rectangle (4,-3);
        \node at(0,3.5){\Rmnum{1}};
        \node at(3,-3.5){\Rmnum{4}};
        \node at(-2.25,-3.5){\Rmnum{4}};
        \node at(0.5,-1.5){\Rmnum{4}};
    \end{tikzpicture}
\end{figure}\par
(1)取如$\Rmnum{1}$所示高斯面,因为导体内部电场为0,两平行板中间电场与高斯面平行,故该高斯面电通量为0。
故其中没有静电荷即
$$
\sigma_2S+\sigma_3S=0\to \sigma_1=-\sigma_2
$$
即两平板相向两面的电荷面密度大小相等符号相反。\par
(2)取如$\Rmnum{2}$所示高斯面,由场强叠加原理知该高斯面左右两侧场强大小相等,方向相反。
故知$\Rmnum{3},\Rmnum{4}$两高斯面电通量相等,故两高斯面内静电荷量相等即
$$\sigma_1S=\sigma_4S\to \sigma_1=\sigma_4$$
即两平板相背两面的电荷面密度大小相等符号相反。\par
(3)$$\left\{\begin{matrix}
    \sigma_1+\sigma_2=3\\
   \sigma_3+\sigma_4=7\\
   \sigma_1=\sigma_4\\
   \sigma_2+\sigma_3=0
   \end{matrix}\right.$$
   解得$$
   \sigma_1=5\mu\mathrm{C/m^2},\sigma_2=-2\mu\mathrm{C/m^2},\sigma_3=2\mu\mathrm{C/m^2},\sigma_4=5\mu\mathrm{C/m^2}
   $$
\end{sol}\par

{\zihao{5}\heiti\color{red} 1-52}
\begin{sol}
(1)
$$U_2=\int_{\infty}^{R_3}\frac{q+Q}{4\pi\varepsilon_0r^2}\d r=\frac{q+Q}{4\pi\varepsilon_0R_3}$$
$$U_1=U_2+\int_{R_2}^{R_1}\frac{q}{4\pi\varepsilon_0r^2}\d r=\frac{q+Q}{4\pi\varepsilon_0R_3}+\frac{q}{4\pi\varepsilon_0}(\frac{1}{R_1}-\frac{1}{R_2})$$
\par
(2)$$\Delta U=U_1-U_2=\frac{q}{4\pi\varepsilon_0}(\frac{1}{R_1}-\frac{1}{R_2})$$\par
(3)$$U_1=U_2=\frac{q+Q}{4\pi\varepsilon_0R_3}$$
$$\Delta U=0$$\par
(4)情形(1):$$
\begin{aligned}
    U_2&=0\\
    U_1&=U_2+\int_{R_2}^{R_1}\frac{q}{4\pi\varepsilon_0r^2}\d r=\frac{q}{4\pi\varepsilon_0}(\frac{1}{R_1}-\frac{1}{R_2})\\
    \Delta U&=\int_{R_2}^{R_1}\frac{q}{4\pi\varepsilon_0r^2}\d r=\frac{q}{4\pi\varepsilon_0}(\frac{1}{R_1}-\frac{1}{R_2})
\end{aligned}
$$
情形(2):$$
\begin{aligned}
    U_2&=0\\
    U_1&=U_2=0\\
    \Delta U&=0
\end{aligned}
$$\par
(5)设平衡后球体所带电荷为$q^{\prime}$则球壳内表面所带电荷为$-q^{\prime}$,外表面所带电荷为$Q+q^{\prime}$则
球壳电势为
$$
U_2=\int_{\infty}^{R_3}\frac{Q+q^{\prime}}{4\pi\varepsilon_0r^2}\d r=\frac{Q+q^{\prime}}{4\pi\varepsilon_0R_3}
$$
则球体电势为$$U_1=U_2+\int_{R_2}^{R_1}\frac{q^{\prime}}{4\pi\varepsilon_0r^2}\d r=\frac{Q+q^{\prime}}{4\pi\varepsilon_0R_3}+\frac{q^{\prime}}{4\pi\varepsilon_0}(\frac{1}{R_1}-\frac{1}{R_2})$$
又因为球体接地,故
$$U_1=0$$
解得
$$q^\prime=\frac{Q}{R_3}(\frac{1}{R_2}-\frac{1}{R_1}-\frac{1}{R_3})$$
于是有
$$
\begin{aligned}
    U_1&=0\\
    U_2&=\frac{1}{4\pi\varepsilon_0}\frac{(R_2-R_1)Q}{R_1R_2+R_2R_3-R_3R_1}\\
    \Delta U&=U_1-U_2=\frac{1}{4\pi\varepsilon_0}\frac{(R_1-R_2)Q}{R_1R_2+R_2R_3-R_3R_1}
\end{aligned}
$$
\end{sol}\par

{\zihao{5}\heiti\color{red} 1-57}
\begin{sol}
(1)
设上极板带电$Q$,则中间导体上表面带电$-Q$,下表面带电$Q$,下极板带电$-Q$,则电容器中间除导体内部的区域的场强为
$$E=\frac{Q}{\varepsilon_0S}$$
则两极板电势差为$$U=E(d-t)$$
故
$$C=\frac{Q}{U}=\frac{\varepsilon_0S}{d-t}$$\par
(2)上面讨论与极板位置无关,故远近无影响。
\end{sol}\par

{\zihao{5}\heiti\color{red} 1-62}
\begin{sol}
(1)
$$\begin{aligned}
    U&=\int_{R_1}^{R_2}\frac{Q}{4\pi\varepsilon_0r^2}\d r+\int_{R_3}^{R_4}\frac{Q}{4\pi\varepsilon_0r^2}\d r\\
    &=\frac{Q}{4\pi\varepsilon_0}(\frac{1}{R_1}-\frac{1}{R_2}+\frac{1}{R_3}-\frac{1}{R_4})
\end{aligned}$$\par
(2)$$
C=\frac{Q}{U}=\frac{4\pi\varepsilon_0}{\frac{1}{R_1}-\frac{1}{R_2}+\frac{1}{R_3}-\frac{1}{R_4}}
$$
\end{sol}\par

{\zihao{5}\heiti\color{red} 附加题}
\begin{sol}
\begin{figure}[!h]
    \begin{tikzpicture}
        \draw[->] (-5,0) --(5,0) node[right]{$x$};
        \draw[->] (0,-5) --(0,5) node[right]{$y$};
        \filldraw (1,1) circle (.1)node[right]{B};
        \filldraw (-1,1) circle (.1)node[left]{A};
        \filldraw (1,-1) circle (.1)node[right]{C};
        \filldraw (-1,-1) circle (.1)node[left]{D};
    \end{tikzpicture}
\end{figure}
因为导线无限长由对称性可知,电荷在导线上均匀分布,设线密度大小为$\lambda$又电场方向向右,故AD带$-\lambda$,BC带$\lambda$。
以A为电势零点则B电势为
$$\begin{aligned}
    U_B&=-E_0a+\frac{-\lambda}{2\pi\varepsilon_0}(\ln r-\ln a)+\frac{-\lambda}{2\pi\varepsilon_0}(\ln a-\ln \sqrt{2}a)+\frac{\lambda}{2\pi\varepsilon_0}(\ln \sqrt{2}a-\ln a)+\frac{\lambda}{2\pi\varepsilon_0}(\ln a-\ln r)\\
    &=-E_0a+\frac{\lambda}{\pi\varepsilon_0}(\ln\sqrt{2}a-\ln r)
\end{aligned}$$
又因为$U_B=U_A=0$
解得$$
\lambda=\frac{\pi\varepsilon_0E_0a}{\ln\sqrt{2}a-\ln r}
$$
故$x$轴上场强分布为
$$\begin{aligned}
    E(x)&=E_0+2\frac{\lambda}{2\pi\varepsilon_0\sqrt{\frac{a^2}{4}+(x-\frac{a}{2})^2}}\frac{x-\frac{a}{2}}{\sqrt{\frac{a^2}{4}+(x-\frac{a}{2})^2}}-2\frac{\lambda}{2\pi\varepsilon_0\sqrt{\frac{a^2}{4}+(x-\frac{a}{2})^2}}\frac{x+\frac{a}{2}}{\sqrt{\frac{a^2}{4}+(x+\frac{a}{2})^2}}\\
    &=E_0\left \{ 1+\frac{a(x-\frac{a}{2}) }{\ln \frac{\sqrt{2}a }{r}[\frac{a^2}{4}+(x-\frac{a}{2} )^2]  } -\frac{a(x+\frac{a}{2}) }{\ln \frac{\sqrt{2}a }{r}[\frac{a^2}{4}+(x+\frac{a}{2} )^2  ]} \right \} 
\end{aligned}$$
代入数值得
$$E(x)=1+\frac{x-0.005}{\ln (100\sqrt{2})[0.25+(x-0.005 )^2]  } -\frac{x+0.005}{\ln (100\sqrt{2})[0.25+(x+0.005 )^2] }\mathrm{V/m}$$
\end{sol}\par
\end{document}