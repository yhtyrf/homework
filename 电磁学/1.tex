\documentclass{phyasgn}
\phyasgn{
  stuname = 姚昊廷,           % 设置学生姓名
  stunum = 22322091,      % 设置学号
  setasgnnum = 1,           % 设置课程次数
  classname = 电磁学,     % 设置课程名称
}

\usepackage{listings}
\lstset{numbers=left,basicstyle=\ttfamily,columns=flexible}



\begin{document}
{\zihao{5}\heiti\color{red} 1-4}
\begin{sol}
对小球受力分析知
$$\frac{F_q}{mg}=\tan\theta$$
又
$$F_q=\frac{q^2}{4\pi\varepsilon_0(2l\sin\theta)^2}$$
故$q=\pm \sqrt{16\pi\tan\theta\sin^2\theta l^2\varepsilon_0 mg}$
\end{sol}\par
{\zihao{5}\heiti\color{red} 1-5}
\begin{sol}
由油滴受力平衡知
$$Eq=mg=\frac{4}{3}\pi r^3\rho g\to q=\frac{4\pi r^3 \rho g}{3E}$$
代入数值得
$$q=-8.03\times 10^{-19}\mathrm{C}$$
\end{sol}\par

{\zihao{5}\heiti\color{red} 1-8}
\begin{sol}
在$(r,\theta)$处电势为
$$\phi(r,\theta)=\frac{q}{4\pi\varepsilon_0}(\frac{1}{\sqrt{r^2+\dfrac{l^2}{4}-rl\cos\theta}}
-\frac{1}{\sqrt{r^2+\dfrac{l^2}{4}+rl\cos\theta}})$$
因为$l\ll r$
,故略去二阶小量$\frac{l^2}{4}$,且运用近似$(1+x)^k= 1+kx(x\ll 1)$可得
$$\phi(r,\theta)=\frac{ql\cos\theta}{4\pi \varepsilon_0r^2}$$
又$\mathbf{E}=-\nabla \phi$,且在极坐标中$\nabla=\frac{\partial }{\partial r}\hat{\mathbf{e}}_r+\frac{\partial }{r \partial\theta}
\hat{\mathbf{e}}_\theta$
故
$$\mathbf{E}(r,\theta)=\frac{ql\cos\theta}{2\pi\varepsilon_0r^3}\hat{\mathbf{e}}_r+\frac{ql\sin\theta}{4\pi\varepsilon_0
r^3}\hat{\mathbf{e}}_\theta$$
其径向和角向分量为
$$
E_r=\frac{ql\cos\theta}{2\pi\varepsilon_0r^3},
E_\theta=\frac{ql\sin\theta}{4\pi\varepsilon_0
r^3}
$$
\end{sol}\par
{\zihao{5}\heiti\color{red} 1-10}
\begin{sol}
(1)其场强大小为
$$E=\frac{q}{4\pi\varepsilon_0}(\frac{1}{(r+l)^2}+\frac{1}{(r-l)^2}-\frac{2}{r^2})$$
泰勒展开并保留二阶余项后得
$$\begin{aligned}
  E&=\frac{q}{4\pi\varepsilon_0r^2}(1-\frac{2l}{r}+\frac{3l^2}{r^2}+1+\frac{2l}{r}+\frac{3l^2}{r^2}-2)\\
&=\frac{6ql^2}{4\pi\varepsilon_0r^4}\\
&=\frac{3Q}{4\pi\varepsilon_0r^4}
\end{aligned}$$\par
(2)$$\begin{aligned}
  U(r)&=\frac{q}{4\pi\varepsilon_0}(\frac{1}{r+l}+\frac{1}{r-l}-\frac{2}{r})\\
  &=\frac{q}{4\pi\varepsilon_0r}\frac{2l^2}{r^2-l^2}
\end{aligned}$$
因为$l\ll r$,故略去二阶小量$l^2$得
$$U(r)=\frac{2ql^2}{4\pi\varepsilon_0r^3}=\frac{Q}{4\pi\varepsilon_0r^3}$$
\end{sol}\par

{\zihao{5}\heiti\color{red} 1-11}
\begin{sol}
  $P$点场强大小为
  $$\begin{aligned}
    E&=\frac{ql}{4\pi\varepsilon_0}\left [(x^2+\frac{l^2}{2}-lx)^{-\frac{3}{2}}-(x^2+\frac{l^2}{2}+lx)^{-\frac{3}{2}}\right ]\\
    &=\frac{ql}{4\pi\varepsilon_0}(x^2+\frac{l^2}{2})^{-\frac{3}{2}}\left[(1-\frac{lx}{x^2+\frac{l^2}{2}})^{-\frac{3}{2}}-(1+\frac{lx}{x^2+\frac{l^2}{2}})^{-\frac{3}{2}}\right]
  \end{aligned}$$
  因为$l\ll x$,故
  $$\begin{aligned}
    (x^2+\frac{l^2}{2})^{-\frac{3}{2}}\left[(1-\frac{lx}{x^2+\frac{l^2}{2}})^{-\frac{3}{2}}-(1+\frac{lx}{x^2+\frac{l^2}{2}})^{-\frac{3}{2}}\right]&=
    (x^2+\frac{l^2}{2})^{-\frac{3}{2}}\left [1+\frac{3lx}{2(x^2+\frac{l^2}{2})}-1+\frac{3lx}{2(x^2+\frac{l^2}{2})}\right ]\\
    &=x^{-3}\frac{3lx}{x^2}\\
    &=\frac{3l}{x^4}\\
  \end{aligned}$$
  故$$
E=\frac{3ql^2}{4\pi\varepsilon_0x^4}
  $$
  方向竖直向上
\end{sol}

\begin{sol}
  (1)
  $$\begin{aligned}
    C&=\frac{\varepsilon_0S}{d}\\
    C'&=\frac{\varepsilon_r\varepsilon_0S}{d}\\
    \Delta Q&=U(C'-C)=\frac{(\varepsilon_r-1U\varepsilon_0S)}{d}
  \end{aligned}$$
  (2)$$\begin{aligned}
    D_1&=D_2\\
    \varepsilon_r\varepsilon_0E_1&=\varepsilon_0E_2\\
    E_1d+E_2d&=U\\
  \end{aligned}$$
  解得
  $$\begin{aligned}
    D=\frac{\varepsilon_r\varepsilon_0U}{d(\varepsilon_r+1)}
  \end{aligned}$$
  又因为$D=\sigma$,故
  $$Q=\sigma S=\frac{\varepsilon_r\varepsilon_0US}{d(\varepsilon_r+1)}$$
  $$\begin{aligned}
    W&=U(Q-C'U)\\
    &=\frac{-\varepsilon_r^2\varepsilon_0U^2S}{(\varepsilon_r+1)d}
  \end{aligned}$$
  (3)$$\begin{aligned}
    C''&=\frac{Q}{U}\\
    &=\frac{\varepsilon_r\varepsilon_0S}{d(\varepsilon_r+1)}\\
    C'''&=\frac{\varepsilon_0S}{2d}\\
  \end{aligned}$$
  $$\begin{aligned}
    A&=\frac{Q^2}{2C'''}-\frac{Q^2}{2C''}\\
    &=\frac{(C''-C''')Q^2}{2C''C'''}\\
    &=\frac{(\frac{\varepsilon_r\varepsilon_0S}{d(\varepsilon_r+1)}-\frac{\varepsilon_0S}{2d})(\frac{\varepsilon_r\varepsilon_0US}{d(\varepsilon_r+1)})^2}{2\frac{\varepsilon_r\varepsilon_0S}{d(\varepsilon_r+1)}\frac{\varepsilon_0S}{2d}}
  \end{aligned}$$
\end{sol}
\end{document}