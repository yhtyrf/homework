\documentclass{phyasgn}
\phyasgn{
  stuname = 姚昊廷,           % 设置学生姓名
  stunum = 22322091,      % 设置学号
  setasgnnum = 13,           % 设置课程次数
  classname = 电磁学,     % 设置课程名称
}

\usepackage{listings}
\usepackage{tikz}
\usepackage{amssymb}
\usepackage{t-angles}
\usepackage{amssymb}
\usepackage{tikz}
\usepackage{mathrsfs}
%\usepackage{autobreak} 
%\usepackage{fixdif} 
\usetikzlibrary{quotes,angles}
\usetikzlibrary{calc}
\usetikzlibrary{decorations.pathreplacing}
\lstset{numbers=left,basicstyle=\ttfamily,columns=flexible}
\makeatletter
\newcommand{\rmnum}[1]{\romannumeral #1}
\newcommand{\Rmnum}[1]{\expandafter\@slowromancap\romannumeral #1@}
\makeatother


\begin{document}

\begin{sol}[3-11]
$$\begin{aligned}
    \mathscr{E}&=\int_a^b\frac{v\mu_0I\d r}{2\pi r}\\
    &=\frac{v\mu_0I}{2\pi}\ln\frac{b}{a}\\
    &=3.7\times 10^{-5}\text{V}
\end{aligned}$$
a端电势高
\end{sol}\par

\begin{sol}[3-13]
    (1)$$\begin{aligned}
        \mathscr{E}&=\int_0^R \omega rB\d r\\
        &=\frac{\omega BR^2 }{2}
    \end{aligned}$$
    (2)从b到a\\
    (3)$$\begin{aligned}
        L&=\int_0^R rBIdr\\
        &=\frac{BIR^2}{2}
    \end{aligned}$$
    方向垂直纸面向里\\
    (4)会\\
    (5)相当于多个电阻并联,感应电动势不变
\end{sol}\par

\begin{sol}[3-30]
    (1)取半径为$r$$(\frac{D_2}{2}<r<\frac{D_1}{2})$的环形回路,
    由对称性知该环路上的磁感应强度均沿切向,则由安培环路定理
    知
    $$
    2\pi r B=\mu_0 NI
    $$
    则
    $$
    B=\frac{\mu_0NI}{2\pi r}
    $$
    $$\begin{aligned}
        \Phi_B&=\int_{\frac{D_2}{2}}^{\frac{D_1}{2}}Bh\d r\\
        &=\int_{\frac{D_2}{2}}^{\frac{D_1}{2}}\frac{\mu_0NI}{2\pi r}h\d r\\
        &=\frac{\mu_0NIh}{2\pi}\ln\frac{D_1}{D_2}
    \end{aligned}$$
    $$\begin{aligned}
        \varPsi &=N\Phi_B\\
        &=\frac{\mu_0N^2Ih}{2\pi}\ln\frac{D_1}{D_2}
    \end{aligned}$$
    故自感系数为
    $$\begin{aligned}
        L&=\frac{\varPsi}{I}\\
        &=\frac{\mu_0N^2h}{2\pi}\ln\frac{D_1}{D_2}
    \end{aligned}$$
    (2)$$L=\frac{4\pi\times 10^{-7}\times 1000\times 1000\times 0.01}{2\pi}\ln\frac{0.2}{0.1}\text{H}=1.4\times 10^{-3}\text{H}$$
\end{sol}\par

\begin{sol}[3-34]
    $$\left\{\begin{matrix}
        L_1+L_2+2M=1\text{H}\\
       L_1+L_2-2M=0.4\text{H}
       \end{matrix}\right.$$
       解得$M=0.15\text{H}$
\end{sol}\par

\begin{sol}[3-35]
    (1)$$B=\frac{\mu_0NI}{2\pi r}+\frac{\mu_0I}{2\pi(d-r)}$$
    $$\begin{aligned}
        \Phi&=\int_{a}^{d-a}B\d r\\
        &=\frac{\mu_0I}{\pi}\ln\frac{d-a}{a}
    \end{aligned}$$
    自感系数$$L=\frac{\Phi}{I}=\frac{\mu_0}{\pi}\ln\frac{d-a}{a}$$
    因为$a\ll d$故
    $$L\approx \frac{\mu_0}{\pi}\ln\frac{d}{a}$$
    故$$L=\frac{4\pi \times 10^{-7}}{\pi}\ln\frac{200}{1}\text{H}=2.1\times10^{-6}\text{H}$$
    (2)$$\begin{aligned}
        A&=\int F\d r\\
        &=\int_{d}^{2d}\frac{\mu_0 I^2}{2\pi r}\d r\\
        &=\frac{\mu_0I^2}{2\pi}\ln2\\
        &=5.5\times10^{-5}\text{J}
    \end{aligned}$$
    (3)$$\begin{aligned}
        \Delta W&=W_2-W_1\\
        &=\frac{L_2I^2}{2}-\frac{L_1I^2}{2}\\
        &\frac{1}{2}(\frac{\mu_0}{\pi}\ln\frac{2d}{a}-\frac{\mu_0}{\pi}\ln\frac{d}{a})\\
        &=\frac{\mu_0I^2}{\pi}\ln2\\
        &=5.5\times10^{-5}\text{J}
    \end{aligned}$$
    能量增加,来自电源
\end{sol}\par
\end{document}