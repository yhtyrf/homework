\documentclass{phyasgn}
\phyasgn{
  stuname = 姚昊廷,           % 设置学生姓名
  stunum = 22322091,      % 设置学号
  setasgnnum = 9,           % 设置课程次数
  classname = 电磁学,     % 设置课程名称
}

\usepackage{listings}
\usepackage{tikz}
\usepackage{amssymb}
\usepackage{t-angles}
\usepackage{amssymb}
\usepackage{tikz}
%\usepackage{autobreak} 
%\usepackage{fixdif} 
\usetikzlibrary{quotes,angles}
\usetikzlibrary{calc}
\usetikzlibrary{decorations.pathreplacing}
\lstset{numbers=left,basicstyle=\ttfamily,columns=flexible}
\makeatletter
\newcommand{\rmnum}[1]{\romannumeral #1}
\newcommand{\Rmnum}[1]{\expandafter\@slowromancap\romannumeral #1@}
\makeatother


\begin{document}
{\zihao{5}\heiti\color{red} 1-66}
\begin{sol}
    (1)$$j=\frac{I}{S}=\sigma E$$故
    $$\begin{aligned}
        E_1&=\frac{I}{\sigma_1 S}\\
        E_2&=\frac{I}{\sigma_2 S}
    \end{aligned}$$
    (2)$$\begin{aligned}
        U_{AB}&=E_1d_1=\frac{Id_1}{\sigma_1 S}\\
        U_{AB}&=E_2d_2=\frac{Id_2}{\sigma_2 S}
    \end{aligned}$$
\end{sol}\par

{\zihao{5}\heiti\color{red} 2-3(思考题)}
\begin{sol}
    $$\begin{aligned}
        \vec{F}_{12}&=\frac{\mu_0}{4\pi}\oint\limits_{(L_1)}\oint\limits_{(L_2)}\frac{I_1I_2\d \vec{l}_1\times (\d \vec{l}_2\times\hat{\mathbf{r}}_{12})}{r_{12}^2}\\
        &=\frac{\mu_0}{4\pi}\oint\limits_{(L_1)}\oint\limits_{(L_2)}\frac{I_1I_2[(\d \vec{l}_1\cdot \hat{\mathbf{r}}_{12})\d \vec{l}_2-(\d \vec{l}_1\cdot \d \vec{l}_2)\hat{\mathbf{r}}_{12}]}{r_{12}^2}\\
        &=-\frac{\mu_0}{4\pi}\oint\limits_{(L_1)}\oint\limits_{(L_2)}\frac{I_1I_2(\d \vec{l}_1\cdot \d \vec{l}_2)\hat{\mathbf{r}}_{12}}{r_{12}^2}
    \end{aligned}$$
    又因为被积函数连续,故积分可交换顺序,即
    $$\vec{F}_{12}=-\frac{\mu_0}{4\pi}\oint\limits_{(L_2)}\oint\limits_{(L_1)}\frac{I_1I_2(\d \vec{l}_1\cdot \d \vec{l}_2)\hat{\mathbf{r}}_{12}}{r_{12}^2}$$
    同理
    $$\vec{F}_{21}=-\frac{\mu_0}{4\pi}\oint\limits_{(L_2)}\oint\limits_{(L_1)}\frac{I_1I_2(\d \vec{l}_1\cdot \d \vec{l}_2)\hat{\mathbf{r}}_{21}}{r_{21}^2}$$
    又因为$\hat{\mathbf{r}}_{21}=-\hat{\mathbf{r}}_{12}$,故
    $$\vec{F}_{12}=-\vec{F}_{21}$$
\end{sol}\par

{\zihao{5}\heiti\color{red} 2-4}
\begin{sol}
    $$B=\frac{\mu_0}{2\pi d}\sqrt{I_1^2+I_2^2}=7.2\times 10^{-5}\text{T}$$
    由两个分量组成,其中
    $$B_1=\frac{\mu_0I_1}{2\pi d}=4.0\times 10^{-5}\text{T}$$
    方向垂直纸面向里
    $$B_2=\frac{\mu_0I_2}{2\pi d}=6.0\times 10^{-5}\text{T}$$
    方向平行纸面向右
\end{sol}\par

{\zihao{5}\heiti\color{red} 2-5}
\begin{sol}
    (1)由系统对称性知,磁感应强度一定沿轴线方向,且三条边贡献相等
    不妨设其中一条边为线段$(0,0,0)\to  (2a,0,0)$且其余两边都在$xOy$平面上
    且均在第一象限则轴线上一点坐标为$(a,\sqrt{3}a,r_0)$,则位于点$(x,0,0)$处的电流元
    在该点产生的磁感应强度为
    $$\begin{aligned}
        \d B&=\frac{\mu_0}{4\pi}\frac{I\d \mathbf{x}\times \hat{\mathbf{r}}}{r^2}\\
        &=\frac{\mu_0I}{4\pi}\frac{-r_0\d x\vec{j}+\frac{\sqrt{3}}{2}a\d x\vec{k}}{\left [ (a-x)^2+\frac{3}{4}a^2+r_0^2  \right ] ^\frac{3}{2}}
    \end{aligned}$$
    我们只取沿轴线即$z$轴方向分量
    则
    $$\begin{aligned}
        \vec{B}&=3\int_0^{2a}\frac{\mu_0I}{4\pi}\frac{\frac{\sqrt{3}}{2}a\vec{k}}{\left [ (a-x)^2+\frac{3}{4}a^2+r_0^2  \right ] ^\frac{3}{2}}\d x\\
        &=\frac{9\mu_0Ia^2}{2\pi(3r_0^2+a^2)\sqrt{3r_0^2+4a^2}}\vec{k}
    \end{aligned}$$
    (2)当$r_0\gg a$时$B=\frac{\sqrt{3}\mu_0Ia^2}{2\pi r_0^3}$,而$S=\sqrt{3}a^2$
    故
    $$B=\frac{\mu_0m}{2\pi r_0^3}$$
\end{sol}\par

{\zihao{5}\heiti\color{red} 2-6}
\begin{sol}
    (1)将载流板分割为无数无限细无限长的载流导线,叠加得
    $$B=\int \d B\cos\theta=\int_{-a}^a\frac{\mu_0}{4\pi}\frac{2(\frac{I}{2a})\d l}{\sqrt{x^2+l^2}}\frac{x}{\sqrt{x^2+l^2}}=\frac{\mu_0I}{2\pi a}\arctan\frac{a}{x}$$
    (2)$$B=\frac{\mu_0\iota }{2}$$
\end{sol}\par
\end{document}