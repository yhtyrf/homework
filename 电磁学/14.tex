\documentclass{phyasgn}
\phyasgn{
  stuname = 姚昊廷,           % 设置学生姓名
  stunum = 22322091,      % 设置学号
  setasgnnum = 14,           % 设置课程次数
  classname = 电磁学,     % 设置课程名称
}

\usepackage{listings}
\usepackage{tikz}
\usepackage{amssymb}
\usepackage{t-angles}
\usepackage{amssymb}
\usepackage{tikz}
\usepackage{mathrsfs}
%\usepackage{autobreak} 
%\usepackage{fixdif} 
\usetikzlibrary{quotes,angles}
\usetikzlibrary{calc}
\usetikzlibrary{decorations.pathreplacing}
\lstset{numbers=left,basicstyle=\ttfamily,columns=flexible}
\makeatletter
\newcommand{\rmnum}[1]{\romannumeral #1}
\newcommand{\Rmnum}[1]{\expandafter\@slowromancap\romannumeral #1@}
\makeatother


\begin{document}

\begin{sol}[5-31]
    连接电源时有
    $$\begin{aligned}
        R_1i_1+L_1\frac{\d i_1}{\d t}+M\frac{\d i_2}{\d t}&=\mathscr{E}\\
        R_2i_2+L_2\frac{\d i_2}{\d t}+M\frac{\d i_1}{\d t}&=0
    \end{aligned}$$
    换成短接时有
    $$\begin{aligned}
        R_1i_1+L_1\frac{\d i_1}{\d t}+M\frac{\d i_2}{\d t}&=0\\
        R_2i_2+L_2\frac{\d i_2}{\d t}+M\frac{\d i_1}{\d t}&=0
    \end{aligned}$$
    由于这两组方程对应的齐次方程相同,故其时间常量相同,下面仅考虑短接时的情况。由短接的
    第一个方程有
    $$\frac{\d i_1}{\d t}=-\frac{M\frac{\d i_2}{\d t}+R_1i_1}{L_1}$$
    代入第二个方程有
    $$L_2\frac{\d i_2}{\d t}-\frac{M^2}{L_1}\frac{\d i_2}{\d t}-\frac{MR_1}{L_1}i_1+R_2i_2=0$$
    引入无漏磁条件$M^2=L_1L_2$有
    $$\begin{aligned}
        -\frac{MR_1}{L_1}i_1+R_2i_2&=0\\
        i_2&=\frac{MR_1}{L_1R_2}i_1\\
        \frac{\d i_2}{\d t}&=\frac{MR_1}{L_1R_2}\frac{\d i_1}{\d t}
    \end{aligned}$$
    代入第一个方程有
    $$\begin{aligned}
        0&=L_1\frac{\d i_1}{\d t}+\frac{M^2R_1}{L_1R_2}\frac{\d i_1}{\d t}+R_1i_1\\
        \frac{\d i_1}{\d t}&=-\frac{R_1R_2}{R_1L_2+R_2L_1}i_1\\
    \end{aligned}$$
    故时间常量为
    $$\tau=\frac{1}{\frac{R_1R_2}{R_1L_2+R_2L_1}}=\frac{R_1L_2+R_2L_1}{R_1R_2}=\frac{L_1}{R_2}+\frac{L_2}{R_1}$$
\end{sol}\par

\begin{sol}[5-32]
    (1)$$\begin{aligned}
        \frac{q}{C}+L\frac{\d i}{\d t}&=0\\
        \frac{q}{C}+L\frac{\d ^2i}{\d t^2}&=0\\
        q&=C_1\sin(\frac{t}{\sqrt{CL}})+C_2\cos(\frac{t}{\sqrt{CL}})
    \end{aligned}$$
    初始条件为$t=0$时,$q=Q$、$i=0$。\par
    故解为
    $$q=Q\cos(\frac{t}{\sqrt{LC}})$$
    欲令线圈磁场能等于电容中电能,即有
    $$\begin{aligned}
        \frac{q^2}{2C}&=\frac{LI^2}{2}\\
        \frac{q^2}{2C}&=\frac{L}{2}(\frac{\d q}{\d t})^2\\
        \frac{Q^2}{2C}\cos^2(\frac{t}{\sqrt{LC}})&=\frac{L}{2}\frac{Q^2}{LC}\cos^2(\frac{t}{\sqrt{LC}})\\
        \tan^2(\frac{t}{\sqrt{LC}})&=1\\
        t&=\frac{\pi}{4}\sqrt{LC}
    \end{aligned}$$
    (2)$$q=Q\cos(\frac{\pi}{4})=\frac{\sqrt{2}}{2}Q$$
\end{sol}\par

\begin{sol}[5-33]
    (1)并联后总电容为$C'=2C=4\mu\text{F}$
    $$\lambda=\frac{R}{2}\sqrt{\frac{C'}{L}}=1.58>1$$
    故不振荡\\
    (2)并联后总电容为$C'=\frac{C^2}{2C}=1\mu\text{F}$
    $$\lambda=\frac{R}{2}\sqrt{\frac{C'}{L}}=0.79<1$$
    故振荡
\end{sol}\par
\end{document}