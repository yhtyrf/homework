\documentclass{phyasgn}
\phyasgn{
  stuname = 姚昊廷,           % 设置学生姓名
  stunum = 22322091,      % 设置学号
  setasgnnum = 11,           % 设置课程次数
  classname = 电磁学,     % 设置课程名称
}

\usepackage{listings}
\usepackage{tikz}
\usepackage{amssymb}
\usepackage{t-angles}
\usepackage{amssymb}
\usepackage{tikz}
%\usepackage{autobreak} 
%\usepackage{fixdif} 
\usetikzlibrary{quotes,angles}
\usetikzlibrary{calc}
\usetikzlibrary{decorations.pathreplacing}
\lstset{numbers=left,basicstyle=\ttfamily,columns=flexible}
\makeatletter
\newcommand{\rmnum}[1]{\romannumeral #1}
\newcommand{\Rmnum}[1]{\expandafter\@slowromancap\romannumeral #1@}
\makeatother


\begin{document}

\begin{sol}[2-21]
(1)取半径为$r$$(\frac{D_2}{2}<r<\frac{D_1}{2})$的环形回路,
由对称性知该环路上的磁感应强度均沿切向,则由安培环路定理
知
$$
2\pi r B=\mu_0 NI
$$
则
$$
B=\frac{\mu_0NI}{2\pi r}
$$
(2)$$\begin{aligned}
    \Phi_B&=\int_{\frac{D_2}{2}}^{\frac{D_1}{2}}Bh\d r\\
    &=\int_{\frac{D_2}{2}}^{\frac{D_1}{2}}\frac{\mu_0NI}{2\pi r}h\d r\\
    &=\frac{\mu_0NIh}{2\pi}\ln\frac{D_1}{D_2}
\end{aligned}$$
\end{sol}\par

\begin{sol}[2-22]
    由对称性知磁感应强度与平面平行且与电流方向垂直,
    取一穿过载流板的矩形回路则由安培环路定理知
    $$B\cdot 2l=\mu_0\iota l$$
    则$$B=\frac{\mu_0\iota}{2}$$
\end{sol}\par

\begin{sol}[2-32]
  \begin{figure}[h]
    \begin{tikzpicture}[scale=0.8]
      \coordinate (p) at (2,-3);
      \coordinate (o) at (0,0);
      \coordinate (a) at (1,0);
      \coordinate (b) at (-1,0);
      \draw[->] (-5,0) --(5,0) node[right]{$x$};
      \draw[->] (0,-5) --(0,5) node[right]{$y$};
      \draw[->] (o) --(p) node[right]{$I_1$};
      \draw[->] (a) --(p) ;
      \draw[->] (b) --(p) ;
      \node at(1,0.2) {$I_2$};
      \node at(-1,0.2) {$-I_2$};
      \pic["$\theta$", draw=red!40, <->, angle eccentricity=0.6, angle radius=0.7cm]
      {angle=p--o--a};
  \end{tikzpicture}
  \end{figure}
  (1)$$\begin{aligned}
    \vec{F}_1&=\frac{\mu_0I_1I_2a}{\pi\sqrt{a^2+b^2-2ab\cos\theta}}\frac{(b\cos\theta-a)\hat{x}-b\sin\theta\hat{y}}{\sqrt{(b\cos\theta-a)^2+b^2\sin^2\theta}}\\
    &=\frac{\mu_0I_1I_2a(b\cos\theta-a)}{\pi(a^2+b^2-2ab\cos\theta)}\hat{x}-\frac{\mu_0I_1I_2ab\sin\theta}{\pi (a^2+b^2-2ab\cos\theta)}\hat{y}
  \end{aligned}$$
  同理$$
  \begin{aligned}
    \vec{F}_2&=\frac{-\mu_0I_1I_2a}{\pi\sqrt{a^2+b^2+2ab\cos\theta}}\frac{(b\cos\theta+a)\hat{x}-b\sin\theta\hat{y}}{\sqrt{(b\cos\theta+a)^2+b^2\sin^2\theta}}\\
    &=\frac{-\mu_0I_1I_2a(b\cos\theta+a)}{\pi(a^2+b^2+2ab\cos\theta)}\hat{x}+\frac{\mu_0I_1I_2ab\sin\theta}{\pi (a^2+b^2+2ab\cos\theta)}\hat{y}
  \end{aligned}$$
  故合力为
  $$\begin{aligned}
    \vec{F}&=\vec{F}_1+\vec{F}_2\\
    &=\frac{\mu_0I_1I_2a}{\pi}\left (\frac{b\cos\theta-a}{a^2+b^2-2ab\cos\theta}-\frac{b\cos\theta+a}{a^2+b^2+2ab\cos\theta} \right )\hat{x}\\
    &+\frac{\mu_0I_1I_2ab\sin\theta}{\pi}\left (\frac{1}{a^2+b^2+2ab\cos\theta}-\frac{1}{a^2+b^2-2ab\cos\theta}\right )\hat{y}
  \end{aligned}$$
  故合力矩为
  $$\begin{aligned}
    \vec{L}&=\vec{r}_1\times\vec{F}_1+\vec{r}_2\times\vec{F}_2\\
    &=a\hat{x}\times\vec{F}_1-a\hat{x}\times\vec{F}_2\\
    &=a\hat{x}\times\left (\vec{F}_1-\vec{F}_2\right )\\
    &=a(F_{1y}-F_{2y})\hat{z}\\
    &=\frac{-\mu_0I_1I_2a^2b\sin\theta}{\pi}\left ( \frac{1}{a^2+b^2+2ab\cos\theta}+\frac{1}{a^2+b^2-2ab\cos\theta}\right )\hat{z}
  \end{aligned}$$
  (2)欲使线圈平衡则$$L=0$$
  即$\sin\theta=0$则
  $$\theta=\left\{\begin{matrix}
    0\\
   \pi
   \end{matrix}\right.$$
   (3)$$\begin{aligned}
    W&=\int_0^{\frac{\pi}{2}} L\d \theta\\
    &=-\frac{\mu_0I_1I_2a}{\pi}\ln\frac{b-a}{b+a}
   \end{aligned}$$
\end{sol}\par

\begin{sol}[2-33]
  由对称性知线圈受力一定垂直于导线方向
  $$\begin{aligned}
    F&=\int \d F\cos\theta\\
    &=2\int_0^\pi\frac{\mu_0I_1I_2\cos\theta\d \theta}{2\pi(l-r\cos\theta)}\\
    &=\mu_0I_1I_2\left ( 1-\frac{l}{\sqrt{l^2-r^2}}\right )
  \end{aligned}$$
\end{sol}\par

\begin{sol}[2-35]
  (1)$$\begin{aligned}
    L_\text{磁}&=NIBS\\
    &=NIabB\\
    &=1.0\times 10^{-6}\text{N}\cdot \text{m}
  \end{aligned}$$
  (2)$$\begin{aligned}
    D&=\frac{L_\text{磁}}{\varphi}\\
    &=1.9\times 10^{-6}\text{N}\cdot \text{m}
  \end{aligned}$$
\end{sol}\par
\end{document}