\documentclass{phyasgn}
\phyasgn{
  stuname = 姚昊廷,           % 设置学生姓名
  stunum = 22322091,      % 设置学号
  setasgnnum = 15,           % 设置课程次数
  classname = 电磁学,     % 设置课程名称
}

\usepackage{listings}
\usepackage{tikz}
\usepackage{amssymb}
\usepackage{t-angles}
\usepackage{amssymb}
\usepackage{tikz}
\usepackage{mathrsfs}
\usepackage{pifont}
%\usepackage{autobreak} 
%\usepackage{fixdif} 
\usetikzlibrary{quotes,angles}
\usetikzlibrary{calc}
\usetikzlibrary{decorations.pathreplacing}
\lstset{numbers=left,basicstyle=\ttfamily,columns=flexible}
\makeatletter
\newcommand{\rmnum}[1]{\romannumeral #1}
\newcommand{\Rmnum}[1]{\expandafter\@slowromancap\romannumeral #1@}
\makeatother


\begin{document}

\begin{sol}[4-25]
    $$M=\frac{m}{V}=\frac{4M}{\pi d^2l}$$
    $$i'=M=\frac{4\times 12000}{\pi(0.025)^2\times0.075}=3.3\times 10^8\text{A/m}$$
\end{sol}\par

\begin{sol}[4-66]
    $$\mathbf{D}=\varepsilon_0\mathbf{E}$$
    电场能量密度为
    $$\rho_E=\frac{\mathbf{D}\cdot \mathbf{E}}{2}=\frac{\varepsilon_0E^2}{2}$$
    因为该电场在该空间处处均匀,故该空间电场能为
    $$W_e=\frac{\varepsilon_0E^2V}{2}=4.43\times 10^{-5}\text{J}$$
    $$\mathbf{H}=\frac{B}{\mu_0}$$
    磁场能量密度为
    $$\rho_B=\frac{\mathbf{B}\cdot \mathbf{H}}{2}=\frac{B^2}{2\mu_0}$$
    因为该磁场在该空间处处均匀,故该空间电场能为
    $$W_B=\frac{B^2V}{2\mu_0}=397.89\text{J}$$
\end{sol}\par

\begin{sol}[4-68]
    由安培环路定理可知该同轴线产生的磁场分布为
    $$B=\left\{\begin{matrix}
        \frac{\mu_0rI}{2\pi a^2}  &0<r<a \\
        \\
         \frac{\mu_0I}{2\pi r} &a<r<b \\
         \\
         \frac{\mu_0I(c^2-r^2)}{2\pi r(c^2-b^2)} &b<r<c \\
         \\
         0&r>c
       \end{matrix}\right.$$
       (1)\\
       \ding{172}导线内,即$0<r<a$处
       $$\begin{aligned}
        B&=\frac{\mu_0rI}{2\pi a^2}\\
        H&=\frac{B}{\mu_0}=\frac{rI}{2\pi a^2}
       \end{aligned}$$
       故该处单位长度能量为
       $$\begin{aligned}
        W&=\iint\frac{\mathbf{B}\cdot \mathbf{H}}{2}\d S\\
        &=\int_0^a\frac{\mu_0I^2r^2}{8\pi^2a^4}2\pi r\d r\\
        &=\frac{\mu_0I^2}{16\pi}
       \end{aligned}$$
       \ding{173}导线和圆筒之间,即$a<r<b$处
       $$\begin{aligned}
        B&=\frac{\mu_0I}{2\pi r}\\
        H&=\frac{B}{\mu_0}=\frac{I}{2\pi r}
       \end{aligned}$$
       故该处单位长度能量为
       $$\begin{aligned}
        W&=\iint\frac{\mathbf{B}\cdot \mathbf{H}}{2}\d S\\
        &=\int_a^b\frac{\mu_0I^2}{8\pi^2r^2}2\pi r\d r\\
        &=\frac{\mu_0I^2}{4\pi}\ln\frac{b}{a}       
       \end{aligned}$$
       \ding{174}圆筒内,即$b<r<c$处
       $$\begin{aligned}
        B&=\frac{\mu_0I(c^2-r^2)}{2\pi r(c^2-b^2)}\\
        H&=\frac{I(c^2-r^2)}{2\pi r(c^2-b^2)}
       \end{aligned}$$
       故该处单位长度能量为
       $$\begin{aligned}
        W&=\iint\frac{\mathbf{B}\cdot \mathbf{H}}{2}\d S\\
        &=\int_b^c\frac{\mu_0I^2(c^2-r^2)^2}{8\pi^2r^2(c^2-b^2)^2}2\pi r\d r\\
        &=\frac{\mu_0I^2}{16\pi(c^2-b^2)^2}(4c^4\ln\frac{c}{b}-3c^4+4b^2c^2-b^4)
       \end{aligned}$$
       \ding{175}圆筒外,即$r>c$处
       $$\begin{aligned}
        B&=0\\
        H&=0
       \end{aligned}$$
       故该处单位长度能量为
       $$\begin{aligned}
        W&=\iint\frac{\mathbf{B}\cdot \mathbf{H}}{2}\d S\\
        &=0
       \end{aligned}$$
       (2)代入数值有
       $$\begin{aligned}
        W_1&=2.5\times10^{-6}\text{J}\\
        W_2&=1.4\times10^{-5}\text{J}\\
        W_3&=6.8\times10^{-7}\text{J}\\
        W_4&=0 
       \end{aligned}$$
       
\end{sol}\par
\end{document}