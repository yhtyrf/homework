\documentclass{phyasgn}
\phyasgn{
  stuname = 姚昊廷,           % 设置学生姓名
  stunum = 22322091,      % 设置学号
  setasgnnum = 16,           % 设置课程次数
  classname = 电磁学,     % 设置课程名称
}

\usepackage{listings}
\usepackage{tikz}
\usepackage{amssymb}
\usepackage{t-angles}
\usepackage{amssymb}
\usepackage{tikz}
\usepackage{mathrsfs}
\usepackage{pifont}
%\usepackage{autobreak} 
%\usepackage{fixdif} 
\usetikzlibrary{quotes,angles}
\usetikzlibrary{calc}
\usetikzlibrary{decorations.pathreplacing}
\lstset{numbers=left,basicstyle=\ttfamily,columns=flexible}
\makeatletter
\newcommand{\rmnum}[1]{\romannumeral #1}
\newcommand{\Rmnum}[1]{\expandafter\@slowromancap\romannumeral #1@}
\makeatother


\begin{document}

\begin{sol}[6-4]
    $$S=\frac{1}{2}E_0H_0$$
    可知$E_0=\sqrt{2S\sqrt{\frac{\mu\mu_0}{\varepsilon\varepsilon_0}}}=1.01\times10^3\text{V/m}$\\
    故$$\sqrt{\overline{E^2}}=\frac{\sqrt{2}}{2}E_0=7.3\times10^2\text{V/m}$$\\
    同理$$\sqrt{\overline{H^2}}=\frac{\sqrt{2}}{2}\sqrt{\frac{\varepsilon_0}{\mu_0}}E_0=1.9\text{A/m}$$
\end{sol}\par

\begin{sol}[6-9]
    (1)$\vec{E}$竖直向下,$\vec{H}$与侧面相切,故$\vec{S}$垂直于侧面\\
    (2)$$\begin{aligned}
        P&=S\cdot 2\pi Rl\\
        &=EH\cdot 2\pi Rl\\
        &=\frac{q}{\varepsilon_0A}\frac{I}{2\pi R}\cdot 2\pi Rl\\
        &=\frac{q}{C}\frac{\d q}{\d t}\\
        &=\frac{\d }{\d t}(\frac{q^2}{2C})
    \end{aligned}$$
\end{sol}\par
\end{document}