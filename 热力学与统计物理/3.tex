\documentclass{phyasgn}
\phyasgn{
  stuname = 姚昊廷,           % 设置学生姓名
  stunum = 22322091,      % 设置学号
  setasgnnum = 3,           % 设置课程次数
  classname = 热力学和统计物理,     % 设置课程名称
}

\usepackage{listings}
\usepackage{tikz}
\usepackage{amssymb}
\usepackage{t-angles}
\usepackage{amssymb}
\usepackage{tikz}
\usepackage{mathrsfs}
\usepackage{pifont}
\usepackage{subfigure}
\usepackage{caption}
\usepackage{float}
\usepackage{mathrsfs}
%\usepackage{autobreak} 
%\usepackage{fixdif} 
\usetikzlibrary{quotes,angles}
\usetikzlibrary{calc}
\usetikzlibrary{decorations.pathreplacing}
\lstset{numbers=left,basicstyle=\ttfamily,columns=flexible}
\makeatletter
\newcommand{\rmnum}[1]{\romannumeral #1}
\newcommand{\Rmnum}[1]{\expandafter\@slowromancap\romannumeral #1@}
\renewcommand{\i}{\mathrm{i}}
\makeatother
\allowdisplaybreaks[4]%允许公式跨页
\renewcommand{\d}{\mathrm{d}}
\usepackage{pdfpages}

\begin{document}

\begin{sol}[1]
    (a)物体自发发生虚变动的条件为
    \begin{align*}
        \d U<T\d S-p\d V\\
    \end{align*}
    此处$S$,$V$不变,故有
    \begin{align*}
        \d U<0
    \end{align*}
    因此若物体达到了内能的极小值,$\d U>0$,不可能自发变化,亦即达到了平衡态。

    (b)\begin{align*}
        \d U&<T\d S-p\d V\\
        \d H&<T\d S+V\d p\\
    \end{align*}
    此处$S$,$p$不变,故有
    \begin{align*}
        \d H<0
    \end{align*}
    因此若物体达到了焓的极小值,$\d H>0$,不可能自发变化,亦即达到了平衡态。

    (c)\begin{align*}
        \d U&<T\d S-p\d V\\
        \d H&<T\d S+V\d p\\
        \d S&>\frac{\d H}{T}-\frac{V\d p}{T}
    \end{align*}
    此处$H$,$p$不变,故有
    \begin{align*}
        \d S>0
    \end{align*}
    因此若物体达到了熵的极大值,$\d S<0$,不可能自发变化,亦即达到了平衡态。

    (d)\begin{align*}
        \d U&<T\d S-p\d V\\
        \d F&<-S\d T-p\d V\\
        -\d F&>S\d T+p\d V\\
        \d T&<\frac{-\d F}{S}-\frac{p}{S}\d V
    \end{align*}
    此处$F$,$V$不变,故有
    \begin{align*}
        \d T<0
    \end{align*}
    因此若物体达到了温度的极小值,$\d T>0$,不可能自发变化,亦即达到了平衡态。

    (e)\begin{align*}
        \d U&<T\d S-p\d V\\
        \d G&<-S\d T+V\d p\\
        -S\d T&>\d G-V\d p\\
        \d T&<\frac{V}{S}\d p-\frac{\d G}{S}
    \end{align*}
    此处$p$,$G$不变,故有
    \begin{align*}
        \d T<0
    \end{align*}
    因此若物体达到了温度的极小值,$\d T>0$,不可能自发变化,亦即达到了平衡态。

    (f)\begin{align*}
        \d U&<T\d S-p\d V\\
        \d V&<\frac{T}{p}\d S-\frac{\d U}{p}
    \end{align*}
    此处$U$,$S$不变,故有
    \begin{align*}
        \d V<0
    \end{align*}
    因此若物体达到了体积的极小值,$\d V>0$,不可能自发变化,亦即达到了平衡态。

    (g)\begin{align*}
        \d U&<T\d S-p\d V\\
        \d F&<-S\d T-p\d V\\
        -\d F&>S\d T+p\d V\\
        \d V&<-\frac{\d F}{p}-\frac{S}{p}\d T
    \end{align*}
    此处$F$,$T$不变,故有
    \begin{align*}
        \d V<0
    \end{align*}
    因此若物体达到了体积的极小值,$\d V>0$,不可能自发变化,亦即达到了平衡态。
\end{sol}\par

\begin{sol}[2]
    \begin{align*}
        \delta S_1&=\frac{\delta U_1}{T_1}+\frac{p_1\delta V_1}{T_1}\\
        \delta S_2&=\frac{\delta U_2}{T_2}+\frac{p_2\delta V_2}{T_2}\\
    \end{align*}
    故
    \begin{align*}
        \delta S&=\delta S_1+\delta S_2\\
        &=\frac{\delta U_1}{T_1}+\frac{p_1\delta V_1}{T_1}+\frac{\delta U_2}{T_2}+\frac{p_2\delta V_2}{T_2}
    \end{align*}
    又因为
    \begin{align*}
        \delta U_1+\delta U_2&=0\\
        \delta V_1+\delta V_2&=0
    \end{align*}
    故得
    \begin{align*}
        T_1&=T_2\\
        p_1&=p_2\\
    \end{align*}
    \begin{align*}
        \delta^2S&=\delta^2S_1+\delta^2S_2\\
        &=\sum_{\alpha=1}^{2}\left[-\frac{C_V^\alpha}{T^2}(\delta T)^2+\frac{1}{T}(\frac{\partial p}{\partial V^\alpha})_T(\delta V^\alpha)^2\right]\\
        &=\sum_{\alpha=1}^{2}n^\alpha\left[-\frac{C_{V,m}^\alpha}{T^2}(\delta T)^2+\frac{1}{T}(\frac{\partial p}{\partial V_m^\alpha})_T(\delta V_m^\alpha)^2\right]\\
    \end{align*}
    故欲使
    \begin{align*}
        \delta^2S<0
    \end{align*}
    又$n^\alpha$是广延量,由于平衡只取决于强度量,即无论$n^\alpha$取何值,该式总成立,故
    \begin{align*}
        -\frac{C_{V,m}^\alpha}{T^2}(\delta T)^2+\frac{1}{T}(\frac{\partial p}{\partial V_m^\alpha})_T(\delta V_m^\alpha)^2<0(\alpha=1,2)
    \end{align*}
    又$\delta V_m^\alpha$与$\delta T$独立,故
    \begin{align*}
        C_{V,m}^\alpha>0,(\frac{\partial p}{\partial V_m^\alpha})_T<0(\alpha=1,2)
    \end{align*}
\end{sol}\par

\begin{sol}[3]
    (1)\begin{align*}
        \d F&=-S\d T-p\d V+\mu\d n\\
    \end{align*}
    故
    \begin{align*}
        \frac{\partial^2 F}{\partial n\partial T}&=\frac{\partial^2 F}{\partial T\partial n}\\
        (\frac{\partial \mu}{\partial T})_{V,n}&=-(\frac{\partial S}{\partial n})_{V,T}
    \end{align*}

    (2)\begin{align*}
        \d G&=-S\d T+V\d p+\mu\d n\\
    \end{align*}
    故
    \begin{align*}
        \frac{\partial^2 G}{\partial n\partial p}&=\frac{\partial^2 G}{\partial p\partial n}\\
        (\frac{\partial \mu}{\partial p})_{T,n}&=(\frac{\partial V}{\partial n})_{p,T}
    \end{align*}
\end{sol}\par

\begin{sol}[4]
    \begin{align*}
        \d U&=T\d S-p\d V+\mu\d n\\
        &=T(\frac{\partial S}{\partial n})_{T,V}\d n+T(\frac{\partial S}{\partial T})_{n,V}\d T+T(\frac{\partial S}{\partial V})_{n,T}\d V-p\d V+\mu\d n
    \end{align*}
    故
    \begin{align*}
        (\frac{\partial U}{\partial n})_{T,V}&=T(\frac{\partial S}{\partial n})_{T,V}+\mu\\
        (\frac{\partial U}{\partial n})_{T,V}-\mu&=T(\frac{\partial S}{\partial n})_{T,V}\\
        (\frac{\partial U}{\partial n})_{T,V}-\mu&=-T(\frac{\partial \mu}{\partial T})_{V,n}
    \end{align*}
\end{sol}\par

\begin{sol}[5]
    平衡相变过程中$T$、$p$不变,故有
    \begin{align*}
        \Delta U_m&=\Delta H_m-p\Delta V_m\\
        \Delta H_m&=L
    \end{align*}
    又由克拉伯龙方程知
    \begin{align*}
        \frac{\d p }{\d T}=\frac{L}{T\Delta V_m}
    \end{align*}
    故
    \begin{align*}
        \Delta V_m=\frac{L\d T}{T\d p}
    \end{align*}
    故
    \begin{align*}
        \Delta U_m&=L-p\frac{L\d T}{T\d p}\\
        &=L\left(1-\frac{p\d T}{T\d p}\right)
    \end{align*}
    对于理想气体至凝聚相的相变过程而言
    \begin{align*}
        \Delta V_m=\frac{RT}{p}
    \end{align*}
    故
    \begin{align*}
        \frac{\d p }{\d T}&=\frac{L}{T\frac{RT}{p}}\\
        &=\frac{Lp}{RT^2}
    \end{align*}
    故
    \begin{align*}
        \Delta U_m
        &=L\left(1-\frac{pRT^2}{TLp}\right)\\
        &=L\left(1-\frac{RT}{L}\right)
    \end{align*}
\end{sol}\par

\begin{sol}[6]
    \begin{align*}
        L&=H_m^\beta-H_m^\alpha\\
        \frac{\d L}{\d T}&=(\frac{\partial H_m^\beta}{\partial T})_p+(\frac{\partial H_m^\beta}{\partial p})_T\frac{\d p}{\d T}-(\frac{\partial H_m^\alpha}{\partial T})_p-(\frac{\partial H_m^\alpha}{\partial p})_T\frac{\d p}{\d T}\\
        &=C_p^\beta-C_p^\alpha+\frac{L}{T(V_m^\beta-V_m^\alpha)}\left[(\frac{\partial H_m^\beta}{\partial p})_T-(\frac{\partial H_m^\alpha}{\partial p})_T\right]\\
        &=C_p^\beta-C_p^\alpha+\frac{L}{T(V_m^\beta-V_m^\alpha)}\left[V_m^\beta-T(\frac{\partial V_m^\beta}{\partial T})_p-V_m^\alpha+T(\frac{\partial V_m^\alpha}{\partial T})_p\right]\\
        &=C_p^\beta-C_p^\alpha+\frac{L}{T}-\frac{L}{V_m^\beta-V_m^\alpha}\left[(\frac{\partial V_m^\beta}{\partial T})_p-(\frac{\partial V_m^\alpha}{\partial T})_p\right]
    \end{align*}
    若$\beta$相是气相$\alpha$相是凝聚相则可略去$V_m^\alpha$及$T(\frac{\partial V_m^\alpha}{\partial T})_p$且$pV_m^\beta=RT$,则
    \begin{align*}
        (\frac{\partial V_m^\beta}{\partial T})_p=\frac{R}{p}
    \end{align*}
    故
    \begin{align*}
        \frac{\d L}{\d T}&=C_p^\beta-C_p^\alpha+\frac{L}{T}-\frac{RL}{pV_m^\beta}\\
        &=C_p^\beta-C_p^\alpha+\frac{L}{T}-\frac{L}{T}\\
        &=C_p^\beta-C_p^\alpha
    \end{align*}
\end{sol}\par

\begin{sol}[7]
    \begin{align*}
        p&=\frac{RT}{V_m-b}-\frac{a}{V_m^2}\\
        (\frac{\partial p}{\partial V_m})_T&=\frac{-RT}{(V_m-b)^2}+\frac{2a}{V_m^3}
    \end{align*}
    等温线极大值点和极小值点满足
    \begin{align*}
        (\frac{\partial p}{\partial V_m})_T&=0\\
        \frac{-RT}{(V_m-b)^2}+\frac{2a}{V_m^3}&=0\\
        \frac{RT}{(V_m-b)^2}&=\frac{2a}{V_m^3}
    \end{align*}
    联立此式与状态方程可得
    \begin{align*}
        p=\frac{2a(V_m-b)}{V_m^3}-\frac{a}{V_m^2}
    \end{align*}
    故
    \begin{align*}
        pV_m^3=a(V_m-2b)
    \end{align*}
    区域一、三满足$(\frac{\partial p}{\partial V_m})_T<0$,虽然化学势较高,但是仍可作为亚稳态以单相存在。C应处于气液不分的临界态,区域二中各点除C外
    均不满足平衡稳定性的要求,因此只能两相共存存在。
\end{sol}\par

\begin{sol}[8]
    \begin{align*}
        \d s&=(\frac{\partial s}{\partial p})_T\d p+(\frac{\partial s}{\partial T})_p\d T\\
        &=-\frac{\partial^2\mu}{\partial T\partial p}\d p+(\frac{\partial s}{\partial T})_p\d T\\
        &=-v\alpha\d p+\frac{c_p}{T}\d T
    \end{align*}
    相变点处
    \begin{align*}
        \d s^{(1)}&=\d s^{(2)}\\
        -v^{(1)}\alpha^{(1)}\d p+\frac{c_p^{(1)}}{T}\d T&=-v^{(2)}\alpha^{(1)}\d p+\frac{c_p^{(2)}}{T}\d T
    \end{align*}
    又因为$v^{(1)}=v^{(2)}=v$,故
    \begin{align*}
        \frac{\d p}{\d T}=\frac{c_p^{(2)}-c^{(1)}_p}{Tv(\alpha^{(2)}-\alpha^{(1)})}
    \end{align*}
    \begin{align*}
        \d v&=(\frac{\partial v}{\partial p})_T\d p+(\frac{\partial v}{\partial T})_p\d T\\
        &=-v\kappa_T\d p+\alpha v\d T
    \end{align*}
    相变点处
    \begin{align*}
        \d v^{(1)}&=\d v^{(2)}\\
        -v^{(1)}\kappa_T^{(1)}\d p+\alpha^{(1)}v^{(1)}\d T&=-v^{(2)}\kappa_T^{(2)}\d p+\alpha^{(2)}v^{(2)}\d T
    \end{align*}
    又因为$v^{(1)}=v^{(2)}=v$,故
    \begin{align*}
        \frac{\d p}{\d T}=\frac{\alpha^{(2)}-\alpha^{(1)}}{\kappa_T^{(2)}-\kappa_T^{(1)}}
    \end{align*}
\end{sol}\par
\end{document}