\documentclass[12pt]{article}
\usepackage{amsmath}
\usepackage{amssymb}
\usepackage[UTF8]{ctex}
\usepackage{geometry}
\geometry{a4paper, left=2.5cm, right=2.5cm, top=2.5cm, bottom=2.5cm}

\title{第一章 热力学的基本规律}
\author{}
\date{}

\begin{document}
\maketitle

\section{热力学系统}
\begin{itemize}
    \item \textbf{定义}:大量微观粒子构成的宏观物质系统
    \item \textbf{分类}:
    \begin{itemize}
        \item 孤立系统(无物质/能量交换)
        \item 闭系(无物质交换,有能量交换)
        \item 开系(有物质/能量交换)
    \end{itemize}
    \item \textbf{特点}:
    \begin{itemize}
        \item 粒子数宏观有限,微观等效无穷大
        \item 粒子持续无规则热运动
    \end{itemize}
\end{itemize}

\section{热力学平衡态}
\begin{itemize}
    \item \textbf{定义}:孤立系统的宏观性质长时间不变
    \item \textbf{关键概念}:
    \begin{itemize}
        \item 驰豫时间(达到平衡的特征时间)
        \item 热动平衡(忽略涨落)
        \item 准热平衡(非孤立系统近似)
    \end{itemize}
\end{itemize}

\section{热平衡定律(第零定律)}
\begin{itemize}
    \item \textbf{内容}:若A与C、B与C热平衡,则A与B热平衡
    \item \textbf{温度定义}:态函数$T$,通过温标量化
    \item \textbf{温标类型}:
    \begin{itemize}
        \item 理想气体温标:
        \begin{equation}
            T = 273.16 \lim_{p_i \to 0} \frac{p}{p_t}
        \end{equation}
        \item 热力学温标(K)
        \item 摄氏温标:$t = T - 273.15$
    \end{itemize}
\end{itemize}

\section{物态方程}
\begin{itemize}
    \item \textbf{基本形式}:
    \begin{equation}
        f(p, V, T) = 0
    \end{equation}
    \item \textbf{关键方程}:
    \begin{itemize}
        \item 理想气体:
        \begin{equation}
            pV = nRT
        \end{equation}
        \item 范德瓦耳斯方程:
        \begin{equation}
            \left( p + \frac{an^2}{V^2} \right)(V - nb) = nRT
        \end{equation}
        \item 昂内斯方程(位力展开):
        \begin{equation}
            P = \frac{nRT}{V} \left[ 1 + \frac{n}{V}B(T) + \cdots \right]
        \end{equation}
    \end{itemize}
    \item \textbf{热力学系数}:
    \begin{align}
        \alpha &= \frac{1}{V} \left( \frac{\partial V}{\partial T} \right)_p \\
        \kappa_T &= -\frac{1}{V} \left( \frac{\partial V}{\partial p} \right)_T \\
        \alpha &= \kappa_T \beta p
    \end{align}
\end{itemize}

\section{准静态过程与功}
\begin{itemize}
    \item \textbf{定义}:无限缓慢的过程,中间态均为平衡态
    \item \textbf{功的计算}:
    \begin{itemize}
        \item 流体:
        \begin{equation}
            W = -\int_{V_A}^{V_B} p \, dV
        \end{equation}
        \item 表面薄膜:
        \begin{equation}
            dW = \sigma \, dA
        \end{equation}
        \item 电介质:
        \begin{equation}
            dW = V E \, dP
        \end{equation}
        \item 磁介质:
        \begin{equation}
            dW = \mu_0 V \mathcal{H} \, d\mathcal{M}
        \end{equation}
    \end{itemize}
    \item \textbf{广义形式}:
    \begin{equation}
        dW = \sum Y_i \, dy_i
    \end{equation}
\end{itemize}

% 后续章节按照相同模式继续添加...

\end{document}