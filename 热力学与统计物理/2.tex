\documentclass{phyasgn}
\phyasgn{
  stuname = 姚昊廷,           % 设置学生姓名
  stunum = 22322091,      % 设置学号
  setasgnnum = 2,           % 设置课程次数
  classname = 热力学和统计物理,     % 设置课程名称
}

\usepackage{listings}
\usepackage{tikz}
\usepackage{amssymb}
\usepackage{t-angles}
\usepackage{amssymb}
\usepackage{tikz}
\usepackage{mathrsfs}
\usepackage{pifont}
\usepackage{subfigure}
\usepackage{caption}
\usepackage{float}
\usepackage{mathrsfs}
%\usepackage{autobreak} 
%\usepackage{fixdif} 
\usetikzlibrary{quotes,angles}
\usetikzlibrary{calc}
\usetikzlibrary{decorations.pathreplacing}
\lstset{numbers=left,basicstyle=\ttfamily,columns=flexible}
\makeatletter
\newcommand{\rmnum}[1]{\romannumeral #1}
\newcommand{\Rmnum}[1]{\expandafter\@slowromancap\romannumeral #1@}
\renewcommand{\i}{\mathrm{i}}
\makeatother
\allowdisplaybreaks[4]%允许公式跨页
\usepackage{pdfpages}

\begin{document}

\begin{sol}[1]
    \begin{align*}
        F&=U-TS\\
        \mathrm{d} F&=-S\d T-p\d V\\
    \end{align*}
    故
    \begin{align*}
        \frac{\partial\frac{\partial F}{\partial T}}{\partial V}&=\frac{\partial\frac{\partial F}{\partial V}}{\partial T}\\
        (\frac{\partial S}{\partial V})_T&=(\frac{\partial p}{\partial T})_p
    \end{align*}
    又$(\frac{\partial p}{\partial T})_p>0$,故$(\frac{\partial S}{\partial V})_T>0$,即温度不变时,气体的熵随体积增加。
\end{sol}\par

\begin{sol}[2]
    以$V,T$为参数有
    \begin{align*}
        \d U&=T\d S-p\d V\\
        &=T(\frac{\partial S}{\partial T})_V\d T+(T(\frac{\partial S}{\partial V})_T-p)\d V
    \end{align*}
    故
    \begin{align*}
        (\frac{\partial U}{\partial V})_T&=T(\frac{\partial S}{\partial V})_T-p=0
    \end{align*}
    以$p,T$为参数
    \begin{align*}
        \d U&=T\d S-p\d V\\
        &=T(\frac{\partial S}{\partial p})_T\d p+T(\frac{\partial S}{\partial T})_p\d T-p(\frac{\partial V}{\partial p})_T\d p-p(\frac{\partial V}{\partial T})_p\d T
    \end{align*}
    故
    \begin{align*}
        (\frac{\partial U}{\partial p})_T&=T(\frac{\partial S}{\partial p})_T-p(\frac{\partial V}{\partial p})_T\\
        &=T(\frac{\partial S}{\partial p})_T-T(\frac{\partial S}{\partial V})_T(\frac{\partial V}{\partial p})_T\\
        &=T(\frac{\partial S}{\partial p})_T-T(\frac{\partial S}{\partial p})_T\\
        &=0
    \end{align*}
\end{sol}\par

\begin{sol}[3]
    \begin{align*}
        \mathrm{d}S&=(\frac{\partial S}{\partial p})_T\d p+(\frac{\partial S}{\partial T})_p\d T\\
        &=(\frac{\partial S}{\partial p})_T\d p+(\frac{\partial S}{\partial T})_p(\frac{\partial T}{\partial p})_V\d p+(\frac{\partial S}{\partial T})_p(\frac{\partial T}{\partial V})_p\d V
    \end{align*}
    故
    \begin{align*}
        (\frac{\partial S}{\partial V})_p&=(\frac{\partial S}{\partial T})_p(\frac{\partial T}{\partial V})_p\\
        &=\frac{C_p}{T}(\frac{\partial T}{\partial V})_p
    \end{align*}
    $T>0$又由稳定性要求$C_p>0$故$(\frac{\partial S}{\partial V})_p$与$\frac{C_p}{T}(\frac{\partial T}{\partial V})_p$正负性相同,题目得证。
\end{sol}\par


\begin{sol}[4]
    \begin{align*}
        \mathrm{d}U&=(\frac{\partial U}{\partial V})_T\d V+(\frac{\partial U}{\partial T})_V\d T\\
        &=(T(\frac{\partial S}{\partial V})_T-p)\d V+C_V\d T
    \end{align*}
    代入
    \begin{align*}
        \mathrm{d}U&=T\d S-p\d V
    \end{align*}
    得
    \begin{align*}
        T\d S&=T(\frac{\partial S}{\partial V})_T\d V+C_V\d T\\
        &=T(\frac{\partial p}{\partial T})_V\d V+C_V\d T
    \end{align*}
    又因为
    \begin{align*}
        (\frac{\partial p}{\partial T})_T(\frac{\partial V}{\partial p})_T(\frac{\partial T}{\partial V})_p&=-1\\
        (\frac{\partial p}{\partial T})_T&=-\frac{(\frac{\partial V}{\partial T})_p}{(\frac{\partial V}{\partial p})_T}\\
        (\frac{\partial p}{\partial T})_T&=\frac{\alpha}{k_T}
    \end{align*}
    \begin{align*}
        T\d S=T(\frac{\partial S}{\partial V})_T\d V+C_V\d T
    \end{align*}
    又绝热过程中$\d S=0$,故
    \begin{align*}
        T(\frac{\partial S}{\partial V})_T\d V+C_V\d T&=0\\
        T\frac{\alpha}{k_T}\d V+C_V\d T&=0\\
        (\frac{\partial T}{\partial V})_S=-\frac{T\alpha}{C_Vk_T}
    \end{align*}
    因为$T>0,\alpha<0,C_V>0,T>0$故
    \begin{align*}
        (\frac{\partial T}{\partial V})_S>0
    \end{align*}
\end{sol}\par

\begin{sol}[5]
    \begin{align*}
        \mathrm{d}S&=(\frac{\partial S}{\partial T})_p\d T+(\frac{\partial S}{\partial p})_T\d p
    \end{align*}
    在绝热过程中$\d S=0$,即
    \begin{align*}
        (\frac{\partial S}{\partial T})_p\d T+(\frac{\partial S}{\partial p})_T\d p=0
    \end{align*}
    故
    \begin{align*}
        (\frac{\partial T}{\partial p})_S=-\frac{(\frac{\partial S}{\partial p})_T}{(\frac{\partial S}{\partial T})_p}=\frac{T(\frac{\partial V}{\partial T})_p}{C_p}
    \end{align*}
    \begin{align*}
        \mathrm{d}H&=(\frac{\partial H}{\partial T})_p\d T+(\frac{\partial H}{\partial p})_T\d p
    \end{align*}
    在节流过程中$\d H=0$,即
    \begin{align*}
        (\frac{\partial H}{\partial T})_p\d T+(\frac{\partial H}{\partial p})_T\d p=0
    \end{align*}
    故
    \begin{align*}
        (\frac{\partial T}{\partial p})_H&=-\frac{(\frac{\partial H}{\partial p})_T}{(\frac{\partial H}{\partial T})_p}\\
        &=\frac{-(\frac{\partial S}{\partial p})_T+V}{T(\frac{\partial S}{\partial T})_p}\\
        &=\frac{-T(\frac{\partial V}{\partial T})_p+V}{C_p}
    \end{align*}
    故
    \begin{align*}
        (\frac{\partial T}{\partial p})_S-(\frac{\partial T}{\partial p})_H=\frac{V}{C_p}>0
    \end{align*}
\end{sol}\par

\begin{sol}[6]
    \begin{align*}
        (\frac{\partial U}{\partial V})_T&=T(\frac{\partial S}{\partial V})_T-p\\
        &=T(\frac{\partial p}{\partial T})_V-p\\
        &=0
    \end{align*}
    \begin{align*}
        pV&=f(T)\\
        p\d V+V\d p&=\frac{\d f}{\d T}\d T\\
        (\frac{\partial p}{\partial T})_V&=\frac{\frac{\d f}{\d T}}{V}
    \end{align*}
    故
    \begin{align*}
        \frac{\frac{\d f}{\d T}}{V}-p&=0\\
        pV&=T\frac{\d f}{\d T}=f(T)
    \end{align*}
    \begin{align*}
        T\frac{\d f}{\d T}&=f\\
        \ln f&=\ln T+C\\
        f&=CT\\
    \end{align*}
    故得
    \begin{align*}
        pV=CT
    \end{align*}
\end{sol}\par

\begin{sol}[7]
    \begin{align*}
        C_V&=T(\frac{\partial S}{\partial T})_V\\
        (\frac{\partial C_V}{\partial V})_T&=T\frac{\partial^2 S}{\partial T\partial V}\\
        &=T\frac{\partial^2 S}{\partial V\partial T}\\
        &=T\frac{(\frac{\partial p}{\partial T})_V}{\partial T}\\
        &=T(\frac{\partial^2 p}{\partial T^2})_V
    \end{align*}
    又由范德瓦尔斯气体状态方程可以知道
    \begin{align*}
        (\frac{\partial p}{\partial T})_V&=\frac{nR}{V}\\
        (\frac{\partial^2 p}{\partial T^2})_V&=\frac{-nR}{V^2}(\frac{\partial V}{\partial T})_V=0
    \end{align*}
    故
    \begin{align*}
        (\frac{\partial C_V}{\partial V})_T=0
    \end{align*}
\end{sol}\par

\begin{sol}[7]
    1mol范德瓦尔斯气体的物态方程为
    \begin{align*}
       (p+\frac{a}{V_m^2})(V_m-b)=RT
    \end{align*}
    由此可得
    \begin{align*}
        p&=\frac{RT}{V_m-b}-\frac{a}{V_m^2}\\
        (\frac{\partial p}{\partial T})_V&=\frac{R}{V_m-b}\\
        T(\frac{\partial p}{\partial T})_V-p&=\frac{a}{V_m^2}
    \end{align*}
    故
    \begin{align*}
        S_{m}=\int\frac{C_{V,m}}{T}\d T+R\ln(V_m-b)+S_{m0}
    \end{align*}
    故
    \begin{align*}
        (\frac{\partial F_m}{\partial T})_V&=-S_{m}=-(\int\frac{C_{V,m}}{T}\d T+R\ln(V_m-b)+S_{m0})\\
        (\frac{\partial F_m}{\partial V})_T&=-p=-(\frac{RT}{V_m-b}-\frac{a}{V_m^2})
    \end{align*}
    故
    \begin{align*}
        F_m&=-\int S_{m}\d T-\int p\d V\\
        &=-\int (\int\frac{C_{V,m}}{T}\d T+R\ln(V_m-b)+S_{m0})\d T-\int (\frac{RT}{V_m-b}-\frac{a}{V_m^2})\d V\\
        &=-\int (\int\frac{C_{V,m}}{T}\d T)\d T-RT\ln(V_m-b)+S_{m0}T-\frac{a}{V_m}+U_{m0}
    \end{align*}
    故
    \begin{align*}
        U_m&=F_m+TS_m\\
        &=-\int (\int\frac{C_{V,m}}{T}\d T)\d T+\int\frac{C_{V,m}}{T}\d T-\frac{a}{V_m}+U_{m0}
    \end{align*}
\end{sol}\par

\begin{sol}[8]
    \begin{align*}
        \mathrm{d} U&=\bar{\d} Q-p\d V\\
        3(p\d V+V\d p)&=\bar{\d} Q-p\d V\\
        \bar{\d}&=4p\d V+3V\d p
    \end{align*}
    又因为$u=3p=aT^4$,故等温过程中$\d p=0$,故
    \begin{align*}
        Q&=\int_{V_1}^{V_2}4p\d V\\
        &=\frac{4}{3}aT^4\ln\frac{V_2}{V_1}\d V\\
        &=\frac{4}{3}aT^4(V_2-V_1)
    \end{align*}
\end{sol}\par

\begin{sol}[9]
    等温过程中$\d p=0$,故$p$为定值。绝热过程满足
    \begin{align*}
        3(p\d V+V\d p)&=-p\d V\\
        4p\d V+3V\d p&=0\\
        4\frac{\d V}{V}+3\frac{\d p}{p}&=0\\
        \d(\ln(p^3V^4))&=0
    \end{align*}
    故绝热过程满足
    \begin{align*}
        p&=CV^{-\frac{4}{3}}
    \end{align*}
    设卡诺循环为$1\to2\to3\to4\to1$
    等温过程吸热为
    \begin{align*}
        Q=\frac{4}{3}aT_1^4(V_2-V_1)
    \end{align*}
    整个循环对外做功为
    \begin{align*}
        W&=\int_{V_1}^{V_2}p_1\d V+\int_{V_2}^{V_3}p_1V_2^{\frac{4}{3}}V^{-\frac{4}{3}}\d V+\int_{V_3}^{V_4}p_3\d V+\int_{V_4}^{V_1}p_3V_4^{\frac{4}{3}}V^{-\frac{4}{3}}\d V\\
        &=p_1(V_2-V_1)+3p_1V_2^{\frac{4}{3}}(\frac{1}{\sqrt[3]{V_2}}-\frac{1}{\sqrt[3]{V_3}})+p_3(V_4-V_3)+3p_3V_4^{\frac{4}{3}}(\frac{1}{\sqrt[3]{V_3}}-\frac{1}{\sqrt[3]{V_4}})
    \end{align*}
    又因为
    \begin{align*}
        p_1&=\frac{aT_1^4}{3}\\
        p_3&=\frac{aT_2^4}{3}
    \end{align*}
    故
    \begin{align*}
        W=\frac{4T_1^4(T_1-T_2)}{3T_1}a(V_2-V_1)
    \end{align*}
    所以效率
    \begin{align*}
        \eta&=\frac{W}{Q}\\
        &=1-\frac{T_2}{T_1}
    \end{align*}
\end{sol}\par

\begin{sol}[10]
    当介质电位移有$\d D$的改变时,外界做功为
    \begin{align*}
        \bar{\d }W=VE\d D
    \end{align*}
    做代换
    \begin{align*}
        p&\to-E\\
        V&\to VD 
    \end{align*}
    由于
    \begin{align*}
        C_p-C_V=T(\frac{\partial p}{\partial T})_V(\frac{\partial V}{\partial T})_p
    \end{align*}
    故有
    \begin{align*}
        C_E-C_D=-VT(\frac{\partial E}{\partial T})_D(\frac{\partial D}{\partial T})_E
    \end{align*}
    又
    \begin{align*}
        (\frac{\partial D}{\partial T})_E&=E\frac{\d \varepsilon}{\d T}\\
        (\frac{\partial E}{\partial T})_D&=-\frac{D}{\varepsilon^2}\frac{\d \varepsilon}{\d T}
    \end{align*}
    故有
    \begin{align*}
        C_E-C_D&=-VT(E\frac{\d \varepsilon}{\d T})(-\frac{D}{\varepsilon^2}\frac{\d \varepsilon}{\d T})\\
        &=VT\frac{D^2}{\varepsilon^3}(\frac{\d \varepsilon}{\d T})^2
    \end{align*}
\end{sol}\par
\end{document}