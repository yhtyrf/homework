\documentclass{phyasgn}
\phyasgn{
  stuname = 姚昊廷,           % 设置学生姓名
  stunum = 22322091,      % 设置学号
  setasgnnum = 15,           % 设置课程次数
  classname = 数学物理方法,     % 设置课程名称
}

\usepackage{listings}
\usepackage{tikz}
\usepackage{amssymb}
\usepackage{t-angles}
\usepackage{amsmath}
\usepackage{tikz}
\usepackage{mathrsfs}
\usepackage{pifont}
\usepackage{subfigure}
\usepackage{caption}
%\usepackage{autobreak} 
%\usepackage{fixdif} 
\usetikzlibrary{quotes,angles}
\usetikzlibrary{calc}
\usetikzlibrary{decorations.pathreplacing}
\lstset{numbers=left,basicstyle=\ttfamily,columns=flexible}
\makeatletter
\newcommand{\rmnum}[1]{\romannumeral #1}
\newcommand{\Rmnum}[1]{\expandafter\@slowromancap\romannumeral #1@}
\newcommand{\res}{\text{res}}
\renewcommand{\L}{\mathcal{L}}
\newcommand{\F}{\mathcal{F}}
\allowdisplaybreaks[4]
\makeatother


\begin{document}

\begin{sol}[1]
\begin{align*}
    \frac{\d^2 g}{\d t^2}-k^2g=\delta(t-x)
\end{align*}
当$t<x$时,$g=a(x)e^{kt}+b(x)e^{-kt}$,
当$t>x$时,$g=c(x)e^{kt}+d(x)e^{-kt}$
代入$g(x,0)=A,\frac{\d g}{\d t}(x,0)=B$可得
\begin{align*}
    a&=\frac{kA+B}{2k}\\
    b&=\frac{kA-B}{2k}
\end{align*} 
又要求函数值连续导函数差1有
\begin{align*}
  \left\{\begin{matrix}
    ae^{kx}+be^{-kx}=ce^{kx}+de^{-kx}\\
    ake^{kx}-bke^{-kx}=cke^{kx}-dke^{-kx}-1
  \end{matrix}\right.
\end{align*}
解得
\begin{align*}
  c&=\frac{2ake^{kx}+1}{2ke^{kx}}\\
  d&=\frac{2bke^{-kx}-1}{2ke^{kx}}
\end{align*} 
故\begin{align*}
  y(t)=\int_{0}^{t}f(x)(ce^{kt}+de^{-kt})\d x+\int_{t}^{\infty}f(x)(ae^{kt}+be^{-kt})\d x
\end{align*}
\end{sol}\par

\begin{sol}[2]
  \begin{align*}
    \frac{\d^2 g}{\d x^2}+k^2g=\delta(x-t)
\end{align*}
当$x<t$时,$g=a_1(t)\sin kx+a_2(t)\cos kx$,
当$x>t$时,$g=b_1(t)\sin kx+b_2(t)\cos kx$
代入$g(0,t)=A,g(1,t)=B$以及连续条件和跃度条件可得
\begin{align*}
  \left\{\begin{matrix}
    a_2=A\\
    b_1\sin k+b_2\cos k=B\\
    (a_1-b_1)\sin kt+(a_2-b_2)\cos kt=0\\
    (a_1-b_1)k\sin kt-k(a_2-b_2)\cos kt=-1\\
  \end{matrix}\right.
\end{align*}
解得
\begin{align*}
  a_1&=\frac{B-A\cos k+\frac{\cos k}{2k\cos kt}}{\sin k}-\frac{1}{2k\sin kt }\\
  a_2&=A\\
  b_1&=\frac{B-A\cos k+\frac{\cos k}{2k\cos kt}}{\sin k}\\
  b_2&=A-\frac{1}{2k\cos kt}
\end{align*} 
故\begin{align*}
  y(x)=\int_{0}^{x}f(t)(b_1\sin kx+b_2\cos kx)\d t+\int_{x}^{1}f(t)(a_1\sin kx+a_2\cos kx)\d t
\end{align*}
\end{sol}\par

\begin{sol}[3]
  \begin{align*}
    \nabla^2G&=-\delta(\vec{r}-\vec{r}_0)\\
    G|_{r=a}=0
\end{align*}
由电像法可知,该方程解为
\begin{align*}
  G(M,M_0)=\frac{1}{4\pi}\left(\frac{1}{r_0^2+r^2-2rr_0\cos\gamma}-\frac{a}{\sqrt{r^2r_0^2+a^4-2a^2rr_0\cos\gamma}}\right)
\end{align*}
故
\begin{align*}
  u(\vec{r})=-\iiint\limits_{\Omega}Gf(\vec{r}_0)\d V_0-\iint\limits_{\Gamma} g(\vec{r}_0)\frac{\partial G}{\partial n_0}\d S_0
\end{align*}
在球面$\Gamma$上$\d S_0$的法线方向与径向相同,故
\begin{align*}
  \frac{\partial G}{\partial n_0}|_{\Gamma}=-\frac{1}{4\pi a}\frac{a^2-r^2}{(a^2+r^2-2ar\cos\gamma)^\frac{3}{2}}
\end{align*}
引入在无穷远处$u=0$的边界条件可消去另一积分,故在球坐标系中
\begin{align*}
  u(r,\theta,\phi)=&-\int_{a}^{\infty}\int_{0}^{\pi}\int_{0}^{2\pi}G(M,M_0)f(r_0,\theta_0,\phi_0)r_0^2\sin\theta_0\d r_0\d\theta_0\d\phi_0\\
  &+\frac{a(a^2-r^2)}{4\pi}\int_{0}^{2\pi}\int_{0}^{\pi}\frac{g(a,\theta_0,\phi_0)}{(a^2+r^2-2ar\cos\gamma)^\frac{3}{2}}\sin\theta_0\d\theta_0\d\phi_0
\end{align*}
其中$\cos\gamma=\sin\theta\sin\theta_0\cos(\phi-\phi_0)+\cos\theta\cos\theta_0$
\end{sol}\par
\end{document}