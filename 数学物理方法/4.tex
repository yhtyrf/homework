\documentclass{phyasgn}
\phyasgn{
  stuname = 姚昊廷,           % 设置学生姓名
  stunum = 22322091,      % 设置学号
  setasgnnum = 4,           % 设置课程次数
  classname = 数学物理方法,     % 设置课程名称
}

\usepackage{listings}
\usepackage{tikz}
\usepackage{amssymb}
\usepackage{t-angles}
\usepackage{amsmath}
\usepackage{tikz}
\usepackage{mathrsfs}
\usepackage{pifont}
\usepackage{subfigure}
\usepackage{caption}
%\usepackage{autobreak} 
%\usepackage{fixdif} 
\usetikzlibrary{quotes,angles}
\usetikzlibrary{calc}
\usetikzlibrary{decorations.pathreplacing}
\lstset{numbers=left,basicstyle=\ttfamily,columns=flexible}
\makeatletter
\newcommand{\rmnum}[1]{\romannumeral #1}
\newcommand{\Rmnum}[1]{\expandafter\@slowromancap\romannumeral #1@}
\renewcommand{\i}{\mathrm{i}}
\newcommand{\res}{\text{res}}
\allowdisplaybreaks[4]
\makeatother


\begin{document}

\begin{sol}[1]
(1)该级数为缺项的幂级数
 $$\rho=\lim\limits_{n\to\infty}\sqrt[n!]{1}=1\to r=\frac{1}{\rho}=1$$
故收敛半径为1

(2)该级数为幂级数,故收敛条件为
 \begin{align*}
    |\frac{z}{1+z}|&<1\\
    \frac{|z|}{|1+z|}&<1\\
    |z|&<|1+z|
\end{align*} 
故收敛条件为
$$\Re (z)>-\frac{1}{2}$$ 
\end{sol}\par
 
\begin{sol}[2]
    (1)设$t=z-n\pi$则
     \begin{align*}
        \sin z&=\sin(t+n\pi)\\
        &=\left\{\begin{matrix}
           \sin t(n\text{为偶数}) \\
           -\sin t(n\text{为奇数})
           \end{matrix}\right.\\
        &=\left\{\begin{matrix}
            \sum_{k=0}^{\infty}\frac{t^{2k+1}}{(2k+1)!}(n\text{为偶数}) \\
            -\sum_{k=0}^{\infty}\frac{t^{2k+1}}{(2k+1)!}(n\text{为奇数})
            \end{matrix}\right.\\
            &=\left\{\begin{matrix}
                \sum_{k=0}^{\infty}\frac{(z-n\pi)^{2k+1}}{(2k+1)!}(n\text{为偶数}) \\
                -\sum_{k=0}^{\infty}\frac{(z-n\pi)^{2k+1}}{(2k+1)!}(n\text{为奇数})
                \end{matrix}\right.\\
    \end{align*} 

    (2)设
     $$\frac{1}{z^2+z+1}=\sum_{k=0}^{\infty}a_kz^k $$
    则
     \begin{align*}
        \sum_{k=0}^{\infty}a_kz^{k+2}+\sum_{k=0}^{\infty}a_kz^{k+1}+\sum_{k=0}^{\infty}a_kz^k=1
    \end{align*} 
    比较两边相同次幂系数可知$a_{3k}=1,a_{3k+1}=-1,a_{3k+2}=0$,故
     $$\frac{1}{z^2+z+1}=\sum_{k=0}^{\infty}z^{3k}-\sum_{k=0}^{\infty}z^{3k+1} $$

    (3)设$t=z+1$,则
     \begin{align*}
        \frac{1}{z^2}&=\frac{1}{(t-1)^2}\\
        &=\frac{\d}{\d t}(\frac{1}{1-t})\\
        &=\frac{\d}{\d t}\sum_{k=0}^{\infty}t^k\\
        &=\sum_{k=0}^{\infty}\frac{\d}{\d t}t^k\\
        &=\sum_{k=0}^{\infty}(k+1)t^k\\
        &=\sum_{k=0}^{\infty}(k+1)(z+1)^k
    \end{align*} 
    
    (4)设$t=\frac{1}{z}$,则
     \begin{align*}
       \ln\frac{1+z}{1-z}&=\ln\frac{1+\frac{1}{t}}{1-\frac{1}{t}}\\
       &=\ln\frac{t+1}{t-1}\\
       &=\ln(t+1)-\i \pi-\ln(1-t)(\text{此处规定单值分支}\ln(-1)=\i \pi)\\
       &=-2\sum_{n=1}^{\infty}\frac{t^{2n-1}}{(2n-1)!}-\i \pi\\
       &=-2\sum_{n=1}^{\infty}\frac{z^{-(2n-1)}}{(2n-1)!}-\i \pi
    \end{align*} 
\end{sol}\par

\begin{sol}[3]
    (1) \begin{align*}
        \frac{1}{z^2-3z+2}&=\frac{1}{(z-1)(z-2)}\\
        &=\frac{1}{z-2}-\frac{1}{z-1}\\
        &=-\frac{1}{2}\frac{1}{1-\frac{z}{2}}-\frac{1}{z}\frac{1}{1-\frac{1}{z}}\\
        &=-\frac{1}{2}\sum_{n=0}^{\infty}\frac{z^n}{2^n}-\frac{1}{z}\sum_{n=0}^{\infty}z^{-n}\\
        &=\sum_{n=0}^{\infty}\frac{-z^n}{2^{n+1}}-\sum_{n=0}^{\infty}z^{-n-1}
    \end{align*} 

    (2) \begin{align*}
        \frac{1}{z^2-3z+2}&=\frac{1}{(z-1)(z-2)}\\
        &=\frac{1}{z-2}-\frac{1}{z-1}\\
        &=\frac{1}{z}\frac{1}{1-\frac{2}{z}}-\frac{1}{z}\frac{1}{1-\frac{1}{z}}\\
        &=\frac{1}{z}\sum_{n=0}^{\infty}2^nz^{-n}-\frac{1}{z}\sum_{n=0}^{\infty}z^{-n}\\
        &=\sum_{n=0}^{\infty}2^nz^{-n-1}-\sum_{n=0}^{\infty}z^{-n-1}
    \end{align*} 

    (3)设$t=z-1$
     \begin{align*}
        \frac{1}{z^2(z-1)}&=\frac{1}{(t+1)^2t}\\
        &=\frac{1}{t}\frac{1}{t+1}\\
        &=-\frac{1}{t}\frac{\d}{\d t}\frac{1}{1+t}\\
        &=-\frac{1}{t}\frac{\d}{\d t}\sum_{n=0}^{\infty}(-1)^nt^n\\
        &=-\frac{1}{t}\sum_{n=0}^{\infty}\frac{\d}{\d t}(-1)^nt^n\\
        &=-\frac{1}{t}\sum_{n=0}^{\infty}(-1)^nnt^{n-1}\\
        &=\sum_{n=0}^{\infty}(-1)^{n+1}nt^{n-2}\\
        &=\sum_{n=1}^{\infty}(-1)^{n+1}nz^{-(n-2)}
    \end{align*} 

    (4)$$z^3e^{\frac{1}{z}}=\sum_{n=0}^{\infty}\frac{z^{-n+3}}{n!} $$
\end{sol}\par

\begin{sol}[4]
    (1)$\cos z$在全平面解析,故$0$为二阶极点。

    (2)$z=0$是可去奇点\\
    $z=k\pi(k\neq 0)$是一阶极点\\
    $z=\infty$是非孤立奇点

    (3)$\ln z|_{z=1}=2n\pi \i$故对于$n=0$的单值分支,1是二阶极点。对$n\neq 0$的单值分支,1是一阶极点。

    (4)$z=\sqrt{k\pi}$是一阶极点\\
    $z=\infty$是非孤立奇点
\end{sol}\par

\begin{sol}[5]
    (1)设$f(z)=(z-z_0)^m\phi(z)$,$\phi(z)$是全平面解析的。则
     $$\frac{f''(z)}{f(z)}=(m-1)m(z-z_0)^{-2}+2m(z-z_0)^{-1}\frac{\phi'}{\phi}+\frac{\phi''}{\phi} $$
    故$\res(\frac{f''(z)}{f(z)},z_0)=2m\frac{\phi'(z_0)}{\phi(z_0)}$

    (2)设$f(z)=(z-z_0)^{-m}\phi(z)$,$\phi(z)$是全平面解析的。则
     $$\frac{f'(z)}{f(z)}=-m(z-z_0)^{-1}+\frac{\phi'}{\phi} $$
    故$\res(\frac{f'(z)}{f(z)},z_0)=-m$
\end{sol}\par

\begin{sol}[6]
    (1)$$ \frac{1}{z^3-z^5}=\frac{1}{z^3(1-z)(1+z)} $$
    故 \begin{align*}
        \res(\frac{1}{z^3-z^5},0)&=\frac{1}{2!}\frac{\d^2}{\d z^2}(z^3\cdot\frac{1}{z^3-z^5})|_{z=0}\\
        &=-1
    \end{align*} 
     \begin{align*}
        \res(\frac{1}{z^3-z^5},1)&=\lim\limits_{z\to1}[(z-1)\frac{1}{z^3-z^5}]\\
        &=\frac{1}{2}
    \end{align*} 
     \begin{align*}
        \res(\frac{1}{z^3-z^5},1)&=\lim\limits_{z\to-1}[(z+1)\frac{1}{z^3-z^5}]\\
        &=\frac{1}{2}
    \end{align*} 

    (2)选定单值分支后只有$0$一个可去奇点,故
     $$\res(\frac{\sqrt{z}}{\sinh\sqrt{z}},0)=0 $$

    (3) \begin{align*}
        \res(\frac{1}{z^2\sin z},0)&=\frac{1}{2!}\lim\limits_{z\to 0}\frac{\d^2}{\d z^2}(z^3\frac{1}{z^2\sin z})\\
        &=\frac{1}{6}
    \end{align*} 
     \begin{align*}
        \res(\frac{1}{z^2\sin z},k\pi)&=\lim\limits_{z\to k\pi}[(z-k\pi)(z^3\frac{1}{z^2\sin z})]\\
        &=\frac{(-1)^k}{k^2\pi^2}
    \end{align*} 

    (4) $$\frac{1-e^{2z}}{z^4}=-\sum_{n=1}^{\infty}\frac{2^nz^{n-4}}{n!} $$
    故 $$\res(\frac{1-e^{2z}}{z^4},0)=-\frac{2^3}{3!}=-\frac{4}{3} $$
\end{sol}\par
\end{document}