\documentclass{phyasgn}
\phyasgn{
  stuname = 姚昊廷,           % 设置学生姓名
  stunum = 22322091,      % 设置学号
  setasgnnum = 11,           % 设置课程次数
  classname = 数学物理方法,     % 设置课程名称
}

\usepackage{listings}
\usepackage{tikz}
\usepackage{amssymb}
\usepackage{t-angles}
\usepackage{amsmath}
\usepackage{tikz}
\usepackage{mathrsfs}
\usepackage{pifont}
\usepackage{subfigure}
\usepackage{caption}
%\usepackage{autobreak} 
%\usepackage{fixdif} 
\usetikzlibrary{quotes,angles}
\usetikzlibrary{calc}
\usetikzlibrary{decorations.pathreplacing}
\lstset{numbers=left,basicstyle=\ttfamily,columns=flexible}
\makeatletter
\newcommand{\rmnum}[1]{\romannumeral #1}
\newcommand{\Rmnum}[1]{\expandafter\@slowromancap\romannumeral #1@}
\newcommand{\res}{\text{res}}
\renewcommand{\L}{\mathcal{L}}
\newcommand{\F}{\mathcal{F}}
\allowdisplaybreaks[4]
\makeatother


\begin{document}

\begin{sol}[1]
    \begin{align*}
      P'_l(x)=\sum_{n=1}^{l}\frac{(l+n)!}{2(n!)^2(l-n)!}\left(\frac{x-1}{2}\right)^{n-1}
    \end{align*}
    故
    \begin{align*}
      P'_l(1)&=\frac{(l+1)!}{2(1!)^2(l-1)!}\\
      &=\frac{l(l+1)}{2}
    \end{align*}
    $l$为奇数时,$P_l(x)$为奇函数,故$P_l(0)=0$,$l=0$时,$P_l(0)=1$,$l$为偶数时,则$P_l(x)$在$0$处展开的常数项为
    \begin{align*}
      P_l(0)&=\frac{(-1)^n(2l-l)!}{2^l(\frac{l}{2})!(l-\frac{l}{2})!(l-l)!}\\
      &=\frac{(-1)^{\frac{l}{2}}l!}{2^l[(\frac{l}{2})!]^2}
    \end{align*}
    故
    \begin{align*}
      P_l(0)&=\left\{\begin{matrix}
        0&(l\text{为奇数})\\
        1&(l=0)\\
        \frac{(-1)^{\frac{l}{2}}l!}{2^l[(\frac{l}{2})!]^2}&(l\text{为偶数})
      \end{matrix}\right.
    \end{align*}
    \begin{align*}
      P'_l(0)=\frac{(-1)^{\frac{l-1}{2}}(l+2)!}{2^l(\frac{l-1}{2})!(\frac{l+1}{2})!}
    \end{align*}
\end{sol}\par

\begin{sol}[2]
  (1)$n=0$时,$P_0(x)=1$,$P_1(x)=x$,故
  \begin{align*}
    P_0(x)=P'_1(x)+P'_0(x)
  \end{align*}
 设$n=m-1$时该命题成立,则有
 \begin{align*}
  \sum_{l=0}^{m-1}(2l+1)P_l(x)=P'_m(x)+P'_{m-1}(x)
 \end{align*}
 又由递推关系$P'_{m+1}(x)-P'_{m-1}(x)=(2m+1)P_m(x)$,故有
 \begin{align*}
  \sum_{l=0}^{m-1}(2l+1)P_l(x)+(2m+1)P_m(x)=P'_m(x)+P'_{m-1}(x)+P'_{m+1}(x)-P'_{m-1}(x)
 \end{align*}
 即$n=m$时该命题也成立,原命题得证。

 (2)\begin{align*}
  xP_m(x)=\frac{(m+1)P_{m+1}(x)+mP_{m-1}(x)}{2m+1}
 \end{align*}
 故
 \begin{align*}
  I&=\frac{m+1}{2m+1}\int_{-1}^{1}P_{m+1}(x)P_{n}(x)\d x+\frac{m}{2m+1}\int_{-1}^{1}P_{m-1}(x)P_{n}(x)\d x\\
  &=\frac{2(m+1)}{(2m+1)(2n+1)}\delta_{m+1,n}+\frac{2m}{(2m+1)(2n+1)}\delta_{m-1,n}
 \end{align*}
\end{sol}\par

\begin{sol}[3]
  (1)\begin{align*}
    I&=\int_{0}^{\pi}2P_n(\cos\theta)\sin\theta\cos\theta\d\theta\\
    &=-2\int_{0}^{\pi}P_n(\cos\theta)\cos\theta\d\cos\theta\\
    &=2\int_{-1}^{1}xP_n(x)\d x\\
    &=2\int_{-1}^{1}\frac{(n+1)P_{n+1}(x)+nP_{n-1}(x)}{2n+1}\d x\\
    &=\frac{2(n+1)}{2n+1}\int_{-1}^{1}P_{n+1}(x)\d x+\frac{2n}{2n+1}\int_{-1}^{1}P_{n-1}(x)\d x\\
    &=\frac{2(n+1)}{2n+1}\int_{-1}^{1}P_{n+1}(x)P_0(x)\d x+\frac{2n}{2n+1}\int_{-1}^{1}P_{n-1}(x)P_0(x)\d x\\
    &=\frac{4(n+1)}{2n+1}\delta_{n+1,0}+\frac{4n}{2n+1}\delta_{n-1,0}
  \end{align*}

  (2)\begin{align*}
    I&=\int_{-1}^{1}(1+x)^k\frac{1}{2^ll!}\frac{\d^l}{\d x^l}\d x\\
    &=\frac{1}{2^ll!}\left[(1+x)^k\frac{d^{l-1}}{\d x^{l-1}}(x^2-1)^l|^1_{-1}-\int_{-1}^{1}\frac{\d (1+x)^k}{\d x}\frac{d^{l-1}}{\d x^{l-1}}(x^2-1)^l{\dx}\right]\\
    &\overset{\text{再进行}l-1\text{次分部积分}}{=}\frac{(-1)^n}{2^ll!}\int_{-1}^{1}(x^2-1)^l\frac{\d^l(1+x)^k}{\d x^l}\d x
  \end{align*}
  当$k<l$时易知$I=0$。\\
  $k\geqslant l$时
  \begin{align*}
    I&=\frac{1}{2^ll!}\int_{-1}^{1}(1-x^2)^l\frac{\d^l(1+x)^k}{\d x^l}\d x\\
    &=\frac{k!}{2^ll!(k-l)!}\int_{-1}^{1}(1-x^2)^l(1+x)^{k-l}\d x\\
    &=\frac{k!}{2^ll!(k-l)!}\int_{-1}^{1}(1-x)^l(1+x)^k\d x\\
  \end{align*}
  设$t=\frac{x+1}{2}$,则
  \begin{align*}
    I&=\frac{k!}{2^ll!(k-l)!}\int_{0}^{1}[2(1-t)]^l(2t)^k\d (2t-1)\\
    &=\frac{k!2^{k+1}}{l!(k-l)!}\int_{0}^{1}(1-t)^lt^k\d t\\
    &=\frac{k!2^{k+1}}{l!(k-l)!}B(l+1,k+1)\\
    &=\frac{k!2^{k+1}}{l!(k-l)!}\frac{k!l!}{(k+l+1)!}\\
    &=\frac{2^{k+1}(k!)^2}{(k-l)!(k+l+1)!}
  \end{align*}
\end{sol}\par

\begin{sol}[4]
  设$f(x)=a_0P_0(x)+a_1P_1(x)+a_2P_2(x)+a_3P_3(x)$,对比系数可得
  方程组\begin{align*}
    \left\{\begin{matrix}
      a_0-\frac{a_2}{2}=1\\
      a_1-\frac{3a_3}{2}=2\\
      \frac{3a_2}{2}=3\\
      \frac{5a_3}{2}=5
    \end{matrix}\right.
  \end{align*}
  解得\begin{align*}
    \left\{\begin{matrix}
      a_0=2\\
      a_1=5\\
      a_2=2\\
      a_3=2
    \end{matrix}\right.
  \end{align*}
  故\begin{align*}
    f(x)=2P_0(x)+5P_1(x)+2P_2(x)+2P_3(x)
  \end{align*}
\end{sol}\par

\begin{sol}[5]
  定解条件为
  \begin{align*}
    \left\{\begin{matrix}
      \frac{1}{r^2}\partial_r(r^2\partial_r u)+\frac{1}{r^2\sin\theta}\partial_\theta(\sin\theta\partial_\theta u)=0\\
      u|_{r=1}=\cos\theta\\
      u|_{r=2}=1+\cos^2\theta
    \end{matrix}\right.
  \end{align*}
  分离变量$u=R(r)\Theta(\theta)$得
  \begin{align*}
    \left\{\begin{matrix}
      r^2R''+2rR'-l(l+1)R=0\\
      \Theta''+\cot\theta\Theta+l(l+1)\Theta=0
    \end{matrix}\right.
  \end{align*}
  通解为
  \begin{align*}
    \left\{\begin{matrix}
      R=A_lr^l+B_lr^{-l-1}\\
      \Theta=P_l(\cos\theta)
    \end{matrix}\right.
  \end{align*}
  则\begin{align*}
    u=\sum_{l=0}^{\infty}P_l(\cos\theta)(A_lr^l+B_lr^{-l-1})
  \end{align*}
  代入边界条件得
  \begin{align*}
    \sum_{l=0}^{\infty}P_l(\cos\theta)(A_l+B_l)=\cos\theta
  \end{align*}
  对比系数得到
  \begin{align*}
    \left\{\begin{matrix}
      A_0+B_0=0\\
    A_1+B_1=1\\
    A_l+B_l=0(l>1)
    \end{matrix}\right.
  \end{align*}
  又有
  \begin{align*}
    \sum_{l=0}^{\infty}P_l(\cos\theta)(A_l2^l+B_l2^{-l-1})=1+\cos^2\theta
  \end{align*}
  对比系数得到
  \begin{align*}
    \left\{\begin{matrix}
      A_0+\frac{B_0}{2}-\frac{1}{2}(4A_2+\frac{B_2}{8})=1\\
    2A_1+\frac{B_1}{4}=0\\
    \frac{3}{2}(4A_2+\frac{B_2}{4})=1\\
    A_l2^l+B_l2^{-l-1}=0(l>2)
    \end{matrix}\right.
  \end{align*}
  解得\begin{align*}
    \left\{\begin{matrix}
      A_0=\frac{5}{3}\\
      B_0=-\frac{2}{3}\\
      A_1=-\frac{1}{7}\\
      B_1=\frac{8}{7}\\
      A_2=\frac{16}{93}\\
      B_2=-\frac{16}{93}\\
      A_l=B_l=0(l>2)
    \end{matrix}\right.
  \end{align*}
  故
  \begin{align*}
    u=\left(\frac{8}{7 r^2}-\frac{r}{7}\right) \cos (\theta )+\frac{1}{2} \left(\frac{16 r^2}{93}-\frac{16}{93 r^3}\right) \left(3 \cos ^2(\theta )-1\right)-\frac{2}{3 r}+\frac{5}{3}
  \end{align*}
\end{sol}\par

\end{document}