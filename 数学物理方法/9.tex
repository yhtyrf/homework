\documentclass{phyasgn}
\phyasgn{
  stuname = 姚昊廷,           % 设置学生姓名
  stunum = 22322091,      % 设置学号
  setasgnnum = 9,           % 设置课程次数
  classname = 数学物理方法,     % 设置课程名称
}

\usepackage{listings}
\usepackage{tikz}
\usepackage{amssymb}
\usepackage{t-angles}
\usepackage{amsmath}
\usepackage{tikz}
\usepackage{mathrsfs}
\usepackage{pifont}
\usepackage{subfigure}
\usepackage{caption}
%\usepackage{autobreak} 
%\usepackage{fixdif} 
\usetikzlibrary{quotes,angles}
\usetikzlibrary{calc}
\usetikzlibrary{decorations.pathreplacing}
\lstset{numbers=left,basicstyle=\ttfamily,columns=flexible}
\makeatletter
\newcommand{\rmnum}[1]{\romannumeral #1}
\newcommand{\Rmnum}[1]{\expandafter\@slowromancap\romannumeral #1@}
\newcommand{\res}{\text{res}}
\renewcommand{\L}{\mathcal{L}}
\newcommand{\F}{\mathcal{F}}
\allowdisplaybreaks[4]
\makeatother


\begin{document}

\begin{sol}[1]
    定解条件为
    \begin{align*}
        \left\{
            \begin{matrix}
                \frac{\partial^2u}{\partial t^2}-a^2\frac{\partial^2u}{\partial x^2}=0\\
                u|_{t=0}=0\\
                \frac{\partial u}{\partial t}|_{t=0}=\frac{I}{\rho}\delta(x-x_0)
            \end{matrix}\right.
    \end{align*}
    则\begin{align*}
        u(x,t)&=\frac{1}{2a}\int_{x-at}^{x+at}\frac{I}{\rho}\delta(x-x_0)\d x\\
        &=\frac{I}{2a\rho}\int_{x-at}^{x+at}\delta(x-x_0)\d x\\
        &=\frac{I}{2a\rho}\left[H(x-x_0+at)-H(x-x_0-at)\right]
    \end{align*}
    其中$H(x)$为阶跃函数。
\end{sol}\par

\begin{sol}[2]
    设
    $$u(x,y,t)=X(x)Y(y)T(t)$$
    由
    $$\frac{\partial^2u}{\partial t^2}-a^2\nabla^2u=0$$
    知
    $$\frac{T''}{T}=a^2\left(\frac{X''}{X}+\frac{Y''}{Y}\right)$$
    设\begin{align*}
        \frac{X''}{X}&=\lambda_1\\
        \frac{Y''}{Y}&=\lambda_2
    \end{align*}
    则关于$X$的定解条件为
    \begin{align*}
        \left\{\begin{matrix}
            X''(x)+\lambda_1X=0\\
            X(0)=0\\
            X(l)=0\\
        \end{matrix}\right.
    \end{align*}
    $\lambda_1=0$时通解为$X=ax+b$,代入边界条件后得$a=0,b=0$,但$X$不能恒为$0$,故舍去。\\
    $\lambda_1\neq 0$时通解为
    $$X=A\sin\sqrt{\lambda_1}x+B\cos\sqrt{\lambda_1}x$$
    代入边界条件后得
    $$\lambda_1=\left(\frac{n\pi}{l}\right)^2$$
    故本征值与本征函数为
    \begin{align*}
        &\lambda_1=\left(\frac{n\pi}{l}\right)^2\\
        &\sin\left(\frac{n\pi}{l}x\right)
    \end{align*}
    同理关于$Y$的本征值与本征函数为
    \begin{align*}
        &\lambda_2=\left(\frac{m\pi}{l}\right)^2\\
        &\sin\left(\frac{m\pi}{l}y\right)
    \end{align*}
    关于$T$的方程为
    $$T''+a^2(\lambda_1+\lambda_2)T=0$$
    通解为
    $$T=A\cos\sqrt{a^2(\lambda_1+\lambda_2)}t+B\sin\sqrt{a^2(\lambda_1+\lambda_2)}t$$
    代入边界条件$T'(0)=0$后得$B=0$,故
    $$u_{mn}=A_{mn}\cos(\frac{a\pi\sqrt{m^2+n^2}}{l}t)\sin\left(\frac{n\pi}{l}x\right)\sin\left(\frac{m\pi}{l}y\right)$$
    故
    $$u=\displaystyle\sum_{m,n}A_{mn}\cos\left(\frac{a\pi\sqrt{m^2+n^2}}{l}t\right)\sin\left(\frac{n\pi}{l}x\right)\sin\left(\frac{m\pi}{l}y\right)$$
    代入初始条件$u|_{t=0}=Axy(l-x)(l-y)$得
    $$\displaystyle\sum_{m,n}A_{mn}\sin\left(\frac{n\pi}{l}x\right)\sin\sin\left(\frac{m\pi}{l}y\right)=Axy(l-x)(l-y)$$
    故\begin{align*}
        A_{mn}&=\frac{A}{l^2}\int_{-l}^{l}x(l-x)\sin\left(\frac{n\pi}{l}x\right)\d x\int_{-l}^{l}y(l-y)\sin\left(\frac{m\pi}{l}y\right)\d y\\
        &=\frac{4Al^4(-1)^{m+n}}{mn\pi^2}
    \end{align*}
    故
    $$u=\displaystyle\sum_{m,n}\frac{4Al^4(-1)^{m+n}}{mn\pi^2}\cos\left(\frac{a\pi\sqrt{m^2+n^2}}{l}t\right)\sin\left(\frac{n\pi}{l}x\right)\sin\left(\frac{m\pi}{l}y\right)$$
\end{sol}\par

\begin{sol}[3]
    设
    $$u(x,y)=X(x)Y(y)$$
    由
    $$\nabla^2u=0$$
    知
    $$\frac{X''}{X}=-\frac{Y''}{Y}=-\lambda$$
    故$X$通解为
    $$X=a\sin\sqrt{\lambda}x+b\cos\sqrt{\lambda}x$$
    代入边界条件$X(0)=0,X(a)=0$得$X$的本征值与本征函数为
    \begin{align*}
        &\lambda=\left(\frac{n\pi}{a}\right)^2\\
        &\sin\left(\frac{n\pi}{a}x\right)
    \end{align*}
    $Y$通解为
    $$Y=A\sinh\sqrt{\lambda}y+B\cosh\sqrt{\lambda}y$$
    代入边界条件$Y(0)=0$得
    $$Y=A\cosh\left(\frac{n\pi}{a}y\right)$$
    故
    $$u_n=A_n\cosh\left(\frac{n\pi}{a}y\right)\sin\left(\frac{n\pi}{a}x\right)$$
    $$u=\displaystyle\sum_{n=1}^{\infty}A_n\cosh\left(\frac{n\pi}{a}y\right)\sin\left(\frac{n\pi}{a}x\right)$$
    代入$u|_{y=b}=T$得
    $$\displaystyle\sum_{n=1}^{\infty}A_n\cosh\left(\frac{n\pi b}{a}\right)\sin\left(\frac{n\pi}{a}x\right)=T$$
    故\begin{align*}
        A_n&=\frac{\frac{2}{a}\int_{0}^{a}T\sin\left(\frac{n\pi}{a}x\right)\d x}{\cosh\left(\frac{n\pi b}{a}\right)}\\
        &=\frac{2T(1-\cos n\pi)}{n\pi\cosh\left(\frac{n\pi b}{a}\right)}\\
        &=\left\{\begin{matrix}
            \frac{4T}{n\pi\cosh\left(\frac{n\pi b}{a}\right)}&n\text{为奇数}\\
            0&n\text{为偶数}
        \end{matrix}\right.
    \end{align*}
    设$n=2m-1$,故
    $$u=\displaystyle\sum_{m=1}^{\infty}\frac{4T}{(2m-1)\pi\cosh\left(\frac{(2m-1)\pi b}{a}\right)}\cosh\left(\frac{(2m-1)\pi}{a}y\right)\sin\left(\frac{(2m-1)\pi}{a}x\right)$$
\end{sol}\par

\begin{sol}[4]
    设
    $$u(x,t)=X(x)T(t)$$
    由
    $$\frac{\partial u}{\partial t}=a^2\frac{\partial^2u}{\partial x^2}$$
    知
    $$\frac{T'}{T}=a^2\frac{X''}{X}=-\lambda$$
    故$X$通解为
    $$X=a\sin\sqrt{\lambda}x+b\cos\sqrt{\lambda}x$$
    代入边界条件$X(0)=0,a^2X''(l)=0$得$X$的本征值与本征函数为
    \begin{align*}
        &\lambda=\left(\frac{an\pi}{l}\right)^2\\
        &\sin\left(\frac{n\pi}{l}x\right)
    \end{align*}
    $T$通解为
    $$T=Ae^{-\lambda t}$$
    故
    $$u=\sum_{n=1}^{\infty}A_ne^{-\left(\frac{an\pi}{l}\right)^2 t}\sin\left(\frac{n\pi}{l}x\right)$$
    代入边界条件$u|_{t=0}=x$得
    $$x=\sum_{n=1}^{\infty}A_n\sin\left(\frac{n\pi}{l}x\right)$$
    故\begin{align*}
        A_n&=\frac{1}{l}\int_{-l}^{l}x\sin\left(\frac{n\pi}{l}x\right)\d x\\
        &=\frac{2 l (-1)^n}{n\pi}
    \end{align*}
    故
    $$u=\sum_{n=1}^{\infty}\frac{2 l (-1)^n}{n\pi}e^{-\left(\frac{an\pi}{l}\right)^2 t}\sin\left(\frac{n\pi}{l}x\right)$$
\end{sol}\par
\end{document}