\documentclass{phyasgn}
\phyasgn{
  stuname = 姚昊廷,           % 设置学生姓名
  stunum = 22322091,      % 设置学号
  setasgnnum = 6,           % 设置课程次数
  classname = 数学物理方法,     % 设置课程名称
}

\usepackage{listings}
\usepackage{tikz}
\usepackage{amssymb}
\usepackage{t-angles}
\usepackage{amsmath}
\usepackage{tikz}
\usepackage{mathrsfs}
\usepackage{pifont}
\usepackage{subfigure}
\usepackage{caption}
%\usepackage{autobreak} 
%\usepackage{fixdif} 
\usetikzlibrary{quotes,angles}
\usetikzlibrary{calc}
\usetikzlibrary{decorations.pathreplacing}
\lstset{numbers=left,basicstyle=\ttfamily,columns=flexible}
\makeatletter
\newcommand{\rmnum}[1]{\romannumeral #1}
\newcommand{\Rmnum}[1]{\expandafter\@slowromancap\romannumeral #1@}
\renewcommand{\i}{\mathrm{i}}
\newcommand{\res}{\text{res}}
\allowdisplaybreaks[4]
\makeatother


\begin{document}

\begin{sol}[1]
\begin{align*}
    \psi(z)&=\frac{\Gamma'(z)}{\Gamma(z)}\\
    &=\dfrac{\int_{0}^{\infty}-\ln te^{-t}t^{1-z}\d t}{\int_{0}^{\infty}e^{-t}t^{1-z}\d t}
\end{align*}
故
\begin{align*}
    \psi(1)&=\dfrac{-\int_{0}^{\infty}e^{-t}\ln t\d t}{\int_{0}^{\infty}e^{-t}\d t}\\
    &=-\int_{0}^{\infty}e^{-t}\ln t\d t\\
    &=-\gamma
\end{align*}
\end{sol}\par

\begin{sol}[2]
    \begin{align*}
        \psi(z+1)&=\psi(z)+\frac{1}{z}\\
        \psi^{(m)}(z+1)&=\psi^{(m)}(z)+\frac{(-1)^mm!}{z^{m+1}}
    \end{align*}
\end{sol}\par

\begin{pf}[3]
    \begin{align*}
       B(a,b)B(a+b,c)&=\frac{\Gamma(a)\Gamma(b)}{\Gamma(a+b)}\frac{\Gamma(a+b)\Gamma(c)}{\Gamma(a+b+c)}\\
       &=\frac{\Gamma(a)\Gamma(b)\Gamma(c)}{\Gamma(a+b+c)}\\
       &=\frac{\Gamma(b)\Gamma(c)}{\Gamma(b+c)}\frac{\Gamma(a)\Gamma(b+c)}{\Gamma(a+b+c)}\\
       &=B(b,c)B(a,b+c)
    \end{align*}
\end{pf}\par

\begin{sol}[4]
    \begin{align*}
        \int_{0}^{1}(1-x^a)^b\d x&=\frac{1}{a}\int_{0}^{1}x^{a(\frac{1}{a}-1)}(1-x^a)^{b}\d x^a\\
        &=\frac{1}{a}B(\frac{1}{a},b+1)
    \end{align*}
    因为$(1-x^2)^n$是偶函数
    故
    \begin{align*}
        I&=2\int_{0}^{1}(1-x^2)^n\d x\\
        &=B(\frac{1}{2},n+1)\\
        &=\frac{\Gamma(\frac{1}{2})\Gamma(n+1)}{\Gamma(\frac{1}{2}+n+1)}\\
        &=\frac{\sqrt{\pi}n!}{\frac{\sqrt{\pi}(2n+1)!!}{2^{n+1}}}\\
        &=\frac{2(2n)!!}{(2n+1)!!}
    \end{align*}
\end{sol}\par

\begin{sol}[5]
    (1)\begin{align*}
        I&=e^{-4}
    \end{align*}
    (2)\begin{align*}
        I&=\int_{-4}^{7}\delta'(t)\cos(t-1)\d t\\
        &=\int_{-4}^{7}\cos(t-1)\d \delta(t)\\
        &=\cos(t-1)\delta(t)|_{-4}^{7}+\int_{-4}^{7}\sin(t-1)\delta(t)\d t\\
        &=\sin(-1)
    \end{align*}
\end{sol}\par

\begin{sol}[6]
    (1)因为$\sin x=0$时$|\cos x|=1$故
    \begin{align*}
        I&=\sum_{n=0}^{\infty}e^{-n\pi}\\
        &=\frac{1}{1-e^{-\pi}}
    \end{align*}
    (2)\begin{align*}
        I&=\int_{0}^{2\pi}\d \theta\int_{0}^{\infty}\frac{r^2\cos^2\theta\delta(r^2-1)}{r^2+1}r\d r\\
        &=\int_{0}^{2\pi}\cos^2\theta\d \theta\int_{0}^{\infty}\frac{r^2\delta(r^2-1)}{r^2+1}r\d r\\
        &=\frac{\pi}{2}\int_{0}^{\infty}\frac{t\delta(t-1)}{t+1}\d t\\
        &=\frac{\pi}{4}
    \end{align*}
\end{sol}\par
\end{document}