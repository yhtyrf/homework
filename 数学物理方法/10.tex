\documentclass{phyasgn}
\phyasgn{
  stuname = 姚昊廷,           % 设置学生姓名
  stunum = 22322091,      % 设置学号
  setasgnnum = 10,           % 设置课程次数
  classname = 数学物理方法,     % 设置课程名称
}

\usepackage{listings}
\usepackage{tikz}
\usepackage{amssymb}
\usepackage{t-angles}
\usepackage{amsmath}
\usepackage{tikz}
\usepackage{mathrsfs}
\usepackage{pifont}
\usepackage{subfigure}
\usepackage{caption}
%\usepackage{autobreak} 
%\usepackage{fixdif} 
\usetikzlibrary{quotes,angles}
\usetikzlibrary{calc}
\usetikzlibrary{decorations.pathreplacing}
\lstset{numbers=left,basicstyle=\ttfamily,columns=flexible}
\makeatletter
\newcommand{\rmnum}[1]{\romannumeral #1}
\newcommand{\Rmnum}[1]{\expandafter\@slowromancap\romannumeral #1@}
\newcommand{\res}{\text{res}}
\renewcommand{\L}{\mathcal{L}}
\newcommand{\F}{\mathcal{F}}
\allowdisplaybreaks[4]
\makeatother


\begin{document}

\begin{sol}[1]
    定解条件为
    \begin{align*}
        \left\{
        \begin{aligned}
            &\frac{\partial}{\partial r}\left( r \frac{\partial u}{\partial r} \right) + \frac{1}{r} \frac{\partial^2 u}{\partial \theta^2} = 0, \\
            &u|_{r=1} =
            \begin{cases}
                1 & (0 < \theta < \pi), \\
                0 & (\pi < \theta < 2\pi).
            \end{cases}\\
            &u(r,\theta)=u(r,\theta+2\pi)
        \end{aligned}
        \right.
    \end{align*}
    设$u=R(r)\Theta(\theta)$,则得到分离变量后的方程
    \begin{align*}
        \left\{\begin{matrix}
            \Theta''+\lambda\Theta=0\\
            r^2R''+rR'-\lambda R=0
        \end{matrix}\right.
    \end{align*}
    对于$\Theta''+\lambda\Theta=0$通解为
    $$\Theta=a\sin\sqrt{\lambda}\theta+b\cos\sqrt{\lambda}\theta$$
    本征值为
    $$\lambda=n^2$$
    将$\lambda=n^2$代入$r^2R''+rR'-\lambda R=0$得到
    \begin{align*}
        R=\left\{\begin{matrix}
            c_0+d_0\ln r&n=0\\
            c_nr^n+d_nr^{-n}&n\neq 0
        \end{matrix}\right.
    \end{align*}
    故
    $$u=C_0+D_0\ln r+\sum_{n=1}^{\infty}\left(a_n\sin n\theta+b_n\cos n\theta\right)\left(c_nr^n+d_nr^{-n}\right)$$
    又因为$r=0$时$u$应有界,故$D_0=0$,$d_n=0$。故
    $$u=C_0+\sum_{n=1}^{\infty}\left(A_n\sin n\theta+B_n\cos n\theta\right)r^n$$
    代入\begin{align*}
        u|_{r=1} =
            \begin{cases}
                1 & (0 < \theta < \pi), \\
                0 & (\pi < \theta < 2\pi).
            \end{cases}
    \end{align*}
    得
    \begin{align*}
        C_0&=\frac{1}{2\pi}\int_{0}^{\pi}\d \theta\\
        &=\frac{1}{2}\\
        A_n&=\frac{1}{\pi}\int_{0}^{\pi}\sin n\theta\d \theta\\
        &=\left\{\begin{matrix}
            \frac{2}{\pi}&n\text{为奇数}\\
            0&n\text{为偶数}
        \end{matrix}\right.\\
        B_n&=\frac{1}{\pi}\int_{0}^{\pi}\cos n\theta\d \theta\\
        &=0
    \end{align*}
    令$n=2m-1$,故
    $$u=\frac{1}{2}+\sum_{m=1}^{\infty}\frac{2}{\pi}\sin[(2m-1)\theta] r^{2m-1}$$
\end{sol}\par

\begin{sol}[2]
    定解条件为
    \begin{align*}
        \left\{\begin{matrix}
            \frac{1}{a^2}\frac{\partial u}{\partial t}=\frac{\partial}{\partial r}\left(r\frac{\partial u}{\partial r}\right)\\
            u|_{r=1}=0\\
            u|_{t=0}=1-r^2
        \end{matrix}\right.
    \end{align*}
    设$u=R(r)T(t)$,则得到分离变量后的方程
    \begin{align*}
        \left\{\begin{matrix}
            T'-\lambda a^2T=0\\
            R'+rR''-\lambda R=0
        \end{matrix}\right.
    \end{align*}
\end{sol}\par

\begin{pf}[3]
    \begin{equation}
        \frac{\d}{\d x}\left[p\frac{\d y}{\d x}\right]-qy+\lambda \rho y=0
        \label{1}
    \end{equation}
    将\ref{1}两边取复共轭得到
    \begin{equation}
        \frac{\d}{\d x}\left[p\frac{\d y^*}{\d x}\right]-qy^*+\lambda^* \rho y^*=0
        \label{2}
    \end{equation}
    $y^*\cdot\text{式}\ref{1}-y\cdot\text{式}\ref{2}$得
    \begin{align*}
        \frac{\d p}{\d x}(yy^{*'}-y^*y')+p(yy^{*''}-y^*y'')&=(\lambda-\lambda^*)\rho y^*y\\
        \frac{\d}{\d x}[p(yy^{*'}-y^*y')]&=(\lambda-\lambda^*)\rho y^*y\\
        \int_{0}^{l}\frac{\d}{\d x}[p(yy^{*'}-y^*y')]\d x&=\int_{0}^{l}(\lambda-\lambda^*)\rho |y|^2\d x\\
        [p(yy^{*'}-y^*y')]|_0^{l}&=(\lambda-\lambda^*)\int_{0}^{l}\rho |y|^2\d x\\
    \end{align*}
    代入边界条件
    \begin{align*}
        \left\{\begin{matrix}
            \alpha_1y(0)+\alpha_2y'(0)=\beta_1y(l)+\beta_2y'(l)=0\\
            \alpha_1y^*(0)+\alpha_2y^{'*}(0)=\beta_1y^*(l)+\beta_2y^{'*}(l)=0\\
        \end{matrix}\right.
    \end{align*}
    得到
    $$(\lambda-\lambda^*)\int_{0}^{l}\rho |y|^2\d x=[p(yy^{*'}-y^*y')]|_0^{l}=0$$
    又因为$\int_{0}^{l}\rho |y|^2\d x$不恒为0,故$\lambda-\lambda^*=0$,即$\lambda$为实数。
\end{pf}

\begin{sol}[4]
    设$y=a_0\sum_{k=1}^{\infty}a_kx^k$,则有
    \begin{align*}
        y''=2a_2+\sum_{k=1}^{\infty}(k+1)(k+2)a_{k+2}x^k
    \end{align*}
    故
    \begin{align*}
        (k+1)(k+2)a_{k+2}+\omega^2a_k=0
    \end{align*}
    可知
    \begin{align*}
        a_{2k}&=(-1)^k\frac{\omega^{2k}}{(2k)!}a_0\\
        a_{2k+1}&=(-1)^k\frac{\omega^{2k}}{(2k+1)!}a_1\\
    \end{align*}
    故
    $$y=a_0\cos\omega x+\frac{a_1}{\omega}\sin\omega x$$
\end{sol}\par

\begin{sol}[5]
    (1)在有限远处$p,q$均解析,令$t=\frac{1}{x}$,则原方程可化为
    \begin{align*}
        \frac{\d^2y}{\d t^2}+\left(\frac{2}{t}+\frac{2}{t^3}\right)\frac{\d y}{\d t}+\frac{2\lambda}{t^4}y
    \end{align*}
    $t=0$时$t\left(\frac{2}{t}+\frac{2}{t^3}\right)$不解析,故无穷远点不为正则奇点。\par
    (2)原方程可化为
    \begin{align*}
        \frac{\d ^2y}{\d x^2}+\frac{1-2x}{2x(1-x)}\frac{\d y}{\d x}+\frac{\lambda+2q-4qx}{4x(1-x)}y=0
    \end{align*}
    故$x=0$是正则奇点,$x=1$是正则奇点。因为$t=0$时$\frac{1}{t}\frac{1-\frac{2}{t}}{\frac{2}{t}(1-\frac{1}{t})}=\frac{t-2}{2(t-1)}$故无穷远点为正则奇点。
\end{sol}\par

\end{document}