\documentclass{phyasgn}
\phyasgn{
  stuname = 姚昊廷,           % 设置学生姓名
  stunum = 22322091,      % 设置学号
  setasgnnum = 2,           % 设置课程次数
  classname = 数学物理方法,     % 设置课程名称
}

\usepackage{listings}
\usepackage{tikz}
\usepackage{amssymb}
\usepackage{t-angles}
\usepackage{amssymb}
\usepackage{tikz}
\usepackage{mathrsfs}
\usepackage{pifont}
\usepackage{subfigure}
\usepackage{caption}
%\usepackage{autobreak} 
%\usepackage{fixdif} 
\usetikzlibrary{quotes,angles}
\usetikzlibrary{calc}
\usetikzlibrary{decorations.pathreplacing}
\lstset{numbers=left,basicstyle=\ttfamily,columns=flexible}
\makeatletter
\newcommand{\rmnum}[1]{\romannumeral #1}
\newcommand{\Rmnum}[1]{\expandafter\@slowromancap\romannumeral #1@}
\renewcommand{\i}{\mathrm{i}}
\makeatother


\begin{document}

\begin{sol}[1]
   (1)$$\begin{aligned}
    \frac{\partial u}{\partial x}&=2x\\
    \frac{\partial u}{\partial y}&=2\\
    \frac{\partial v}{\partial x}&=2x\\
    \frac{\partial v}{\partial y}&=2y\\
   \end{aligned}$$
   由柯西黎曼条件知该函数在$(1,1)$处可导,全平面不解析。\par
   (2)$$\begin{aligned}
    \frac{\partial u}{\partial x}&=2x\\
    \frac{\partial u}{\partial y}&=-2y\\
    \frac{\partial v}{\partial x}&=2y\\
    \frac{\partial v}{\partial y}&=2x\\
   \end{aligned}$$
   由柯西黎曼条件知该函数在全平面可导,全平面解析。\par
   (3)$$\begin{aligned}
    \frac{\partial u}{\partial x}&=2y\\
    \frac{\partial u}{\partial y}&=2x\\
    \frac{\partial v}{\partial x}&=2x\\
    \frac{\partial v}{\partial y}&=2y\\
   \end{aligned}$$
   由柯西黎曼条件知该函数在$x=0$线上可导,全平面不解析。\par
   (4)$$\begin{aligned}
    \frac{\partial u}{\partial x}&=y^2\\
    \frac{\partial u}{\partial y}&=2xy\\
    \frac{\partial v}{\partial x}&=2xy\\
    \frac{\partial v}{\partial y}&=x^2\\
   \end{aligned}$$
   由柯西黎曼条件知该函数在$(0,0)$处可导,全平面不解析。\par
\end{sol}\par

\begin{sol}[2]
    (1)$$\begin{aligned}
        \frac{\partial v}{\partial x}&=-\frac{\partial u}{\partial y}=\cos x\sinh y\\
        \frac{\partial v}{\partial y}&=\frac{\partial u}{\partial x}=-\sin x\cosh y\\
    \end{aligned}$$
    取积分路径$(0,0)\to(x,0)\to(x,y)$,则虚部为
    $$\begin{aligned}
        v&=\int_{(0,0)}^{(x,y)}\frac{\partial v}{\partial x}\d x+\frac{\partial v}{\partial y}\d y\\
        &=\int_0^x\cos x\sinh 0\d x-\int_0^y\sin x\cosh y\d y+C\\
        &=-\sin x\sinh y+C'
    \end{aligned}$$
    故
    $$f(z)=\cos x\cosh y+\i (-\sin x\sinh y+C')$$\par
    (2)$$\begin{aligned}
        \frac{\partial u}{\partial x}&=\frac{\partial v}{\partial y}=\frac{x}{x^2+y^2}\\
        \frac{\partial u}{\partial y}&=-\frac{\partial v}{\partial x}=\frac{y}{x^2+y^2}\\
    \end{aligned}$$
    取积分路径$(1,1)\to(x,1)\to(x,y)$,则虚部为
    $$\begin{aligned}
        u&=\int_{(1,1)}^{(x,y)}\frac{\partial u}{\partial x}\d x+\frac{\partial u}{\partial y}\d y\\
        &=\int_1^x\frac{1}{x \left(\frac{y^2}{x^2}+1\right)}\d x+\int_1^y\frac{y}{x^2 \left(\frac{y^2}{x^2}+1\right)}\d y+C\\
        &=\frac{1}{2}\ln(x^2+y^2)+C'
    \end{aligned}$$
    故
    $$f(z)=\frac{1}{2}\ln(x^2+y^2)+C'+\i (\arctan\frac{y}{x})$$
 \end{sol}\par

 \begin{sol}[3]
    (1)$$\begin{aligned}
        \frac{\partial v}{\partial x}&=-\frac{\partial u}{\partial y}=\frac{-2y}{x^2+y^2}\\
        \frac{\partial v}{\partial y}&=\frac{\partial u}{\partial x}=\frac{2x}{x^2+y^2}\\
    \end{aligned}$$
    取积分路径$(1,1)\to(x,1)\to(x,y)$,则虚部为
    $$\begin{aligned}
        v&=\int_{(1,1)}^{(x,y)}\frac{\partial v}{\partial x}\d x+\frac{\partial v}{\partial y}\d y\\
        &=\int_1^x\frac{-2y}{x^2+y^2}\d x-\int_1^y\frac{2x}{x^2+y^2}\d y+C\\
        &=2\arctan\frac{y}{x}+C'
    \end{aligned}$$
    故
    $$f(z)=\ln(x^2+y^2)+\i (2\arctan\frac{y}{x}+C')$$\par
    (2)$$\begin{aligned}
        \frac{\partial u}{\partial x}&=\frac{\partial v}{\partial y}=6x^2-6xy-6y^2\\
        \frac{\partial u}{\partial y}&=-\frac{\partial v}{\partial x}=-(3x^2+12xy-3y^2)\\
    \end{aligned}$$
    取积分路径$(0,0)\to(x,0)\to(x,y)$,则虚部为
    $$\begin{aligned}
        u&=\int_{(0,0)}^{(x,y)}\frac{\partial u}{\partial x}\d x+\frac{\partial u}{\partial y}\d y\\
        &=\int_0^x6x^2\d x-\int_0^y(3x^2+12xy-3y^2)\d y+C\\
        &=2x^3-3x^2y-6xy^2+y^3+C'
    \end{aligned}$$
    故
    $$f(z)=2x^3-3x^2y-6xy^2+y^3+C'+\i (x^3+6x^2y-3xy^2-2y^3)$$
 \end{sol}\par
\end{document}