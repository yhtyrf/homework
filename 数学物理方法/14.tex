\documentclass{phyasgn}
\phyasgn{
  stuname = 姚昊廷,           % 设置学生姓名
  stunum = 22322091,      % 设置学号
  setasgnnum = 14,           % 设置课程次数
  classname = 数学物理方法,     % 设置课程名称
}

\usepackage{listings}
\usepackage{tikz}
\usepackage{amssymb}
\usepackage{t-angles}
\usepackage{amsmath}
\usepackage{tikz}
\usepackage{mathrsfs}
\usepackage{pifont}
\usepackage{subfigure}
\usepackage{caption}
%\usepackage{autobreak} 
%\usepackage{fixdif} 
\usetikzlibrary{quotes,angles}
\usetikzlibrary{calc}
\usetikzlibrary{decorations.pathreplacing}
\lstset{numbers=left,basicstyle=\ttfamily,columns=flexible}
\makeatletter
\newcommand{\rmnum}[1]{\romannumeral #1}
\newcommand{\Rmnum}[1]{\expandafter\@slowromancap\romannumeral #1@}
\newcommand{\res}{\text{res}}
\renewcommand{\L}{\mathcal{L}}
\newcommand{\F}{\mathcal{F}}
\allowdisplaybreaks[4]
\makeatother


\begin{document}

\begin{sol}[1]
设$u=R(r)T(t)$,有
\begin{align*}
    T'-m^2T=0
\end{align*}
代入边界条件后可解得
\begin{align*}
    T=A_me^{m^2 t}
\end{align*}
关于$R$的方程为
\begin{align*}
    rR''+R'+\frac{\beta-m^2}{\alpha}rR=0
\end{align*}
代入边界条件后可解得
\begin{align*}
    R=j_0(\sqrt{\frac{\beta-m^2}{\alpha}}r)
\end{align*}
代入$R(a)=0$可得
\begin{align*}
    \sqrt{\frac{\beta-m^2}{\alpha}}=\frac{x_m^{\frac{1}{2}}}{a}\to a=\frac{x_m^{\frac{1}{2}}\sqrt{\alpha}}{\sqrt{\beta-m^2}}
\end{align*}
又因为$j_0(x)=\frac{\sin x}{x}$,故$x_m^{\frac{1}{2}}=n\pi$,故
\begin{align*}
    a=\frac{n\pi\sqrt{\alpha}}{\sqrt{\beta-m^2}}
\end{align*}
又因为$m^2\ge 0$,故
\begin{align*}
    a\ge \frac{\pi\sqrt{\alpha}}{\sqrt{\beta-m^2}}\ge\frac{\pi\sqrt{\alpha}}{\sqrt{\beta}}
\end{align*}
\end{sol}\par

\begin{sol}[2]
记$\hat{x}(p)=\L(x(t)),\hat{f}(p)=\L(f(t))$对方程两边进行拉普拉斯变换后得
\begin{align*}
    p^2\hat{x}-px(0)-x'(0)+2\gamma[p\hat{x}-x(0)]+\omega_0^2\hat{x}&=\hat{f}\\
    (p^2+2\gamma p+\omega_0^2)\hat{x}-p\phi-\psi-2\gamma\phi&=\hat{f}\\
    \hat{x}&=\frac{\hat{f}+p\phi+\psi+2\gamma\phi}{p^2+2\gamma p+\omega_0^2}\\
    \hat{x}&=\frac{\hat{f}+p\phi+\psi+2\gamma\phi}{(p+\gamma)^2+\omega_0^2-\gamma^2}\\
    \hat{x}&=\L\left(\frac{e^{-\gamma t}\sin\sqrt{\omega^2-\gamma^2}t}{\sqrt{\omega^2-\gamma^2}}\right)(\hat{f}+p\phi+\psi+2\gamma\phi)\\
\end{align*}
故
\begin{align*}
    \L(x(t))&=\L\left(\frac{e^{-\gamma t}\sin\sqrt{\omega^2-\gamma^2}t}{\sqrt{\omega^2-\gamma^2}}*f(t)\right)+\phi\L\left((\frac{e^{-\gamma t}\sin\sqrt{\omega^2-\gamma^2}t}{\sqrt{\omega^2-\gamma^2}})'\right)+(\psi+2\gamma\phi)\L(\frac{e^{-\gamma t}\sin\sqrt{\omega^2-\gamma^2}t}{\sqrt{\omega^2-\gamma^2}})\\
    x(t)&=\int_0^tf(\tau)\frac{e^{-\gamma (t-\tau)}\sin\sqrt{\omega^2-\gamma^2} (t-\tau)}{\sqrt{\omega^2-\gamma^2}}\d\tau+\phi\frac{e^{\gamma  (-t)} \cos \left(t \sqrt{\omega_0^2-\gamma ^2}\right)}{\sqrt{\omega_0^2-\gamma ^2}}\\
    &-\frac{\phi\gamma  e^{\gamma  (-t)} \sin \left(t \sqrt{\omega_0^2-\gamma ^2}\right)}{\omega_0^2-\gamma ^2}+(\psi+2\gamma\phi)\frac{e^{-\gamma t}\sin\sqrt{\omega^2-\gamma^2}t}{\sqrt{\omega^2-\gamma^2}}
\end{align*}
\end{sol}\par

\begin{sol}[3]
记$\tilde{u}(\omega,t)=\F(u(x,t))$对方程两边进行傅里叶变换后得
\begin{align*}
   \i\omega\frac{\partial\tilde{u}}{\partial t}&=-\omega^2\tilde{u}\\
   \frac{\partial\tilde{u}}{\partial t}&=\i\omega\tilde{u}\\
   \tilde{u}&=\phi(\omega)e^{\i\omega t}
\end{align*}
又
\begin{align*}
    \tilde{u}(\omega,0)&=\F(e^{-|x|})\\
    &=\frac{2}{\omega^2+1}
\end{align*}
故
\begin{align*}
    \tilde{u}&=\frac{2}{\omega^2+1}e^{\i\omega t}\\
    &=\F(e^{-|x+t|})
\end{align*}
故
\begin{align*}
    u=e^{-|x+t|}
\end{align*}
\end{sol}\par

\begin{sol}[4]
记$\hat{u}(x,p)=\L(u(x,t)),\hat{f}(p)=\L(f(t))$对方程两边进行拉普拉斯变换后得
\begin{align*}
    p^2\hat{u}-pu(x,0)-\frac{\partial u}{\partial t}(x,0)-a^2\frac{\partial^2\hat{u}}{\partial x^2}&=\hat{f}\\
    p^2\hat{u}-a^2\frac{\partial^2\hat{u}}{\partial x^2}&=\hat{f}\\
\end{align*}
$u(\infty,p)$有界,故
\begin{align*}
    \hat{u}=\phi(p)e^{-\frac{p}{a}x}+\frac{\hat{f}}{p^2}
\end{align*}
又
\begin{align*}
    \hat{u}(0,p)=\phi(p)+\frac{\hat{f}}{p^2}=0\to \phi(p)=-\frac{\hat{f}}{p^2}
\end{align*}
故
\begin{align*}
    \hat{u}=-\frac{\hat{f}}{p^2}e^{-\frac{p}{a}x}+\frac{\hat{f}}{p^2}
\end{align*}
故
\begin{align*}
    u&=\int_0^t\int_{0}^{\tau}f(x)\d x\d \tau-\int_{0}^{t-\frac{x}{a}}\int_{0}^{\tau}f(x)\d x\d\tau\\
    &=\int_{t-\frac{x}{a}}^t\int_{0}^{\tau}f(x)\d x\d \tau
\end{align*}
\end{sol}\par

\end{document}