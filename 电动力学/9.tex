\documentclass{phyasgn}
\phyasgn{
  stuname = 姚昊廷,           % 设置学生姓名
  stunum = 22322091,      % 设置学号
  setasgnnum = 9,           % 设置课程次数
  classname = 电动力学,     % 设置课程名称
}

\usepackage{listings}
\usepackage{tikz}
\usepackage{amssymb}
\usepackage{t-angles}
\usepackage{amssymb}
\usepackage{tikz}
\usepackage{mathrsfs}
\usepackage{pifont}
\usepackage{subfigure}
\usepackage{caption}
\usepackage{float}
%\usepackage{autobreak} 
%\usepackage{fixdif} 
\usetikzlibrary{quotes,angles}
\usetikzlibrary{calc}
\usetikzlibrary{decorations.pathreplacing}
\lstset{numbers=left,basicstyle=\ttfamily,columns=flexible}
\makeatletter
\newcommand{\rmnum}[1]{\romannumeral #1}
\newcommand{\Rmnum}[1]{\expandafter\@slowromancap\romannumeral #1@}
\renewcommand{\i}{\mathrm{i}}
\makeatother
\allowdisplaybreaks[4]%允许公式跨页


\begin{document}

\begin{sol}[1]
    (1)\begin{align*}
        \vec{A}_1&=\vec{A}_0\cos\left(k_1z-\omega_1 t\right)\\
        \vec{A}_2&=\vec{A}_0\cdot\cos\left(k_2z-\omega_2 t\right)
    \end{align*}
    故
    \begin{align*}
        \vec{A}_1+\vec{A}_2&=\vec{A}_0\cdot2\left(\cos\left(k_1z-\omega_1 t\right)+\cos\left(k_2z-\omega_2 t\right)\right)\\
        &=\vec{A}_0\cdot2\cos\frac{\left(k_1-k_2\right)z-\left(\omega_1-\omega_2\right)t}{2}\cos\frac{\left(k_1+k_2\right)z-\left(\omega_1+\omega_2\right)t}{2}
    \end{align*}
    又
    \begin{align*}
        k_1&=k+\d k\\
        k2&=k-\d k\\
        \omega_1&=\omega+\d \omega\\
        \omega_2&=\omega+\d \omega
    \end{align*}
    故
    \begin{align*}
        \vec{A}&=\vec{A}_0\cdot2\cos\left(\d k\cdot z-\d \omega\cdot t\right)\cos\left(k\cdot z-\omega\cdot t\right)\\
        &=\vec{A}_0\cdot2\cos\left(\d k\cdot z-\d \omega\cdot t\right)e^{\i \left(kz-\omega t\right)}
    \end{align*}
    (2)相速度:
    \begin{align*}
        kx-\omega t&=0\\
        v_p&=\frac{\omega}{k}
    \end{align*}
    群速度:
    \begin{align*}
        \d kx-\d \omega t&=0\\
        v_g&=\frac{\d \omega}{\d k}
    \end{align*}
\end{sol}

\begin{sol}[2]
    (1)
    \begin{align*}
        \nabla\cdot\vec{B}&=0\\
        \i \vec{k}\cdot\vec{B}&=0\\
        \vec{k}\cdot\vec{B}&=0\\
        \nabla\cdot\vec{D}&=0\\
        \i \vec{k}\cdot\vec{D}&=0\\
        \vec{k}\cdot\vec{D}&=0\\
        \nabla\times\vec{E}&=-\frac{\partial \vec{B}}{\partial t}\\
        \i\vec{k}\times\vec{E}&=\i\omega\vec{B}\\
        \vec{B}&=\frac{\vec{k}\times\vec{E}}{\omega}\\
        \vec{B}\cdot\vec{E}&=0\\
        \nabla\times\vec{H}&=\frac{\partial \vec{D}}{\partial t}\\
        \i\vec{k}\times\vec{H}&=-\i\omega\vec{D}\\
        \vec{D}&=\frac{-\vec{k}\times\vec{B}}{\omega\mu}
    \end{align*}
    故
    \begin{align*}
        \vec{k}\cdot\vec{B}=\vec{k}\cdot\vec{B}=\vec{B}\cdot\vec{D}=\vec{B}\cdot\vec{D}=\vec{B}\cdot\vec{E}=0
    \end{align*}
    \begin{align*}
        \nabla\cdot\vec{E}&=\frac{\rho}{\varepsilon_0}\\
        \i\vec{k}\cdot\vec{E}&=\frac{\rho}{\varepsilon_0}\\
        \vec{k}\cdot\vec{E}&=\frac{\rho}{\i\varepsilon_0}
    \end{align*}
    而介质中由于极化电荷的存在$\rho$一般不为0。故一般$\vec{k}\cdot\vec{E}\neq0$。

    (2)\begin{align*}
        \vec{D}&=\frac{-\vec{k}\times\vec{B}}{\omega\mu}\\
        &=\frac{-\vec{k}\times\frac{\vec{k}\times\vec{E}}{\omega}}{\omega\mu}\\
        &=\frac{-\vec{k}\times(\vec{k}\times\vec{E})}{\omega^2\mu}\\
        &=\frac{k^2\vec{E}-(\vec{k}\cdot\vec{E})\vec{k}}{\omega^2\mu}
    \end{align*}

    (3)\begin{align*}
        \vec{S}&=\vec{E}\times\vec{H}\\
        &=\vec{E}\times\frac{\vec{k}\times\vec{E}}{\mu\omega}\\
        &=\frac{E^2\vec{k}-(\vec{k}\vec{E})\vec{E}}{\mu\omega}
    \end{align*}
    若要令$\vec{k}$与$\vec{S}$在同一方向那么就要使$(\vec{k}\vec{E})\vec{E}=0$,一般不满足。
\end{sol}

\begin{sol}[3]
    \begin{align*}
        A_x&=A_0\cos\left(kz-\omega t\right)\\
        A_y&=A_0\cos\left(kz-\omega t+\frac{\pi}{2}\right)
    \end{align*}
    故
    \begin{align*}
        A_x^2+A_y^2=A_0^2
    \end{align*}
    即圆偏振
\end{sol}

\begin{sol}[4]
    \begin{align*}
        \vec{A}_1&=\left(a\cos\left(kz-\omega t\right),a\cos\left(kz-\omega t+\frac{\pi}{2}\right)\right)\\
        \vec{A}_2&=\left(b\cos\left(kz-\omega t\right),-b\cos\left(kz-\omega t+\frac{\pi}{2}\right)\right)
    \end{align*}
    故
    \begin{align*}
        \vec{A}&=\left(a\cos\left(kz-\omega t\right)+b\cos\left(kz-\omega t\right),a\cos\left(kz-\omega t+\frac{\pi}{2}\right)-b\cos\left(kz-\omega t+\frac{\pi}{2}\right)\right)\\
        &=\left((a+b)\cos\left(kz-\omega t\right),(a-b)\cos\left(kz-\omega t+\frac{\pi}{2}\right)\right)\\
    \end{align*}
    若$\frac{a}{b}=\pm 1$,则线偏振。\\
    若$\frac{a}{b}\neq\pm 1$,则椭圆偏振。
\end{sol}
\end{document}