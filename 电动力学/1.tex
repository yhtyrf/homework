\documentclass{phyasgn}
\phyasgn{
  stuname = 姚昊廷,           % 设置学生姓名
  stunum = 22322091,      % 设置学号
  setasgnnum = 1,           % 设置课程次数
  classname = 电动力学,     % 设置课程名称
}

\usepackage{listings}
\usepackage{tikz}
\usepackage{amssymb}
\usepackage{t-angles}
\usepackage{amssymb}
\usepackage{tikz}
\usepackage{mathrsfs}
\usepackage{pifont}
\usepackage{subfigure}
\usepackage{caption}
%\usepackage{autobreak} 
%\usepackage{fixdif} 
\usetikzlibrary{quotes,angles}
\usetikzlibrary{calc}
\usetikzlibrary{decorations.pathreplacing}
\lstset{numbers=left,basicstyle=\ttfamily,columns=flexible}
\makeatletter
\newcommand{\rmnum}[1]{\romannumeral #1}
\newcommand{\Rmnum}[1]{\expandafter\@slowromancap\romannumeral #1@}
\renewcommand{\i}{\mathrm{i}}
\makeatother


\begin{document}

\begin{sol}[1]
    (1)\begin{align*}
        \text{右边}&=b_i\partial _ia_j+a_i\partial _ib_j-\varepsilon_{klm}b_l\varepsilon_{ijk}\partial _ia_j-\varepsilon_{klm}a_l\varepsilon_{ijk}\partial _ib_j\\
        &=b_i\partial _ia_j+a_i\partial _ib_j+(\delta_{im}\delta_{jl}-\delta_{il}\delta_{jm})b_l\partial _ia_j+(\delta_{im}\delta_{jl}-\delta_{il}\delta_{jm})a_l\partial _ib_j\\
        &=b_i\partial _ia_j+a_i\partial _ib_j+b_j\partial _ma_j-b_i\partial _ia_m+a_j\partial _mb_j-a_i\partial _ib_m\\
        &=b_j\partial _ma_j+a_j\partial _mb_j\\
        &=\text{左边}
    \end{align*}
    (2)由(1)知$\nabla(\vec{A}\cdot\vec{A})=2\vec{A}\cdot\nabla\vec{A}+2\vec{A}\times(\nabla\times\vec{A})$
故
$$\vec{A}\times(\nabla\times\vec{A})=\frac{1}{2}\nabla(\vec{A}\cdot\vec{A})-\vec{A}\cdot\nabla\vec{A}$$
\end{sol}\par

\begin{sol}[2]
    (1)\begin{align*}
        \nabla r&=\frac{x-x'}{\sqrt{(x-x')^2+(y-y')^2+(z-z')^2}}\vec{e}_x+\frac{y-y'}{\sqrt{(x-x')^2+(y-y')^2+(z-z')^2}}\vec{e}_y\\
        &+\frac{z-z'}{\sqrt{(x-x')^2+(y-y')^2+(z-z')^2}}\vec{e}_z\\
        &=\frac{(x-x')\vec{e}_x+(y-y')\vec{e}_y+(z-z')\vec{e}_z}{\sqrt{(x-x')^2+(y-y')^2+(z-z')^2}}\\
        &=\frac{\vec{r}}{r}
    \end{align*}
    \begin{align*}
        \nabla' r&=\frac{x'-x}{\sqrt{(x-x')^2+(y-y')^2+(z-z')^2}}\vec{e}_x+\frac{y'-y}{\sqrt{(x-x')^2+(y-y')^2+(z-z')^2}}\vec{e}_y\\
        &+\frac{z'-z}{\sqrt{(x-x')^2+(y-y')^2+(z-z')^2}}\vec{e}_z\\
        &=\frac{(x'-x)\vec{e}_x+(y'-y)\vec{e}_y+(z'-z)\vec{e}_z}{\sqrt{(x-x')^2+(y-y')^2+(z-z')^2}}\\
        &=-\frac{\vec{r}}{r}
    \end{align*}
    \begin{align*}
        \nabla \frac{1}{r}&=-\frac{x-x'}{\left((x-x')^2+(y-y')^2+(z-z')^2\right)^{3/2}}\vec{e}_x-\frac{y-y'}{\left((x-x')^2+(y-y')^2+(z-z')^2\right)^{3/2}}\vec{e}_y\\
        &-\frac{z-z'}{\left((x-x')^2+(y-y')^2+(z-z')^2\right)^{3/2}}\vec{e}_x\vec{e}_z\\
        &=\frac{(x'-x)\vec{e}_x+(y'-y)\vec{e}_y+(z'-z)\vec{e}_z}{\left((x-x')^2+(y-y')^2+(z-z')^2\right)^{3/2}}\\
        &=-\frac{\vec{r}}{r^3}
    \end{align*}
    \begin{align*}
        \nabla '\frac{1}{r}&=\frac{x-x'}{\left((x-x')^2+(y-y')^2+(z-z')^2\right)^{3/2}}\vec{e}_x+\frac{y-y'}{\left((x-x')^2+(y-y')^2+(z-z')^2\right)^{3/2}}\vec{e}_x\\
        &+\frac{z-z'}{\left((x-x')^2+(y-y')^2+(z-z')^2\right)^{3/2}}\vec{e}_x\vec{e}_z\\
        &=-\frac{(x'-x)\vec{e}_x+(y'-y)\vec{e}_y+(z'-z)\vec{e}_z}{\left((x-x')^2+(y-y')^2+(z-z')^2\right)^{3/2}}\\
        &=\frac{\vec{r}}{r^3}
    \end{align*}
    \begin{align*}
        \nabla \times\frac{\vec{r}}{r^3}&=(\nabla \frac{1}{r^3})\times \vec{r}+\frac{1}{r^3}(\nabla\times r)\\
        &=(\nabla \frac{1}{r^3})\times \vec{r}\\
        &=-3\frac{\vec{r}}{r^5}\times r\\
        &=0
    \end{align*}
    \begin{align*}
        \nabla \cdot\frac{\vec{r}}{r^3}&=(\nabla \frac{1}{r^3})\cdot\vec{r}+\frac{1}{r^3}(\nabla\cdot r)\\
        &=(\nabla \frac{1}{r^3})\cdot\vec{r}+\frac{3}{r^3}\\
        &=-3\frac{\vec{r}}{r^5}\cdot r+\frac{3}{r^3}\\
        &=0
    \end{align*}
    \begin{align*}
        \nabla' \cdot\frac{\vec{r}}{r^3}&=(\nabla' \frac{1}{r^3})\cdot\vec{r}+\frac{1}{r^3}(\nabla'\cdot r)\\
        &=(\nabla' \frac{1}{r^3})\cdot\vec{r}-\frac{3}{r^3}\\
        &=3\frac{\vec{r}}{r^5}\cdot r-\frac{3}{r^3}\\
        &=0
    \end{align*}
    (2)\begin{align*}
        \nabla\cdot \vec{r}&=3
    \end{align*}
    \begin{align*}
        \nabla\times \vec{r}&=0
    \end{align*}
    \begin{align*}
        (\vec{a}\cdot\nabla)\times \vec{r}&=0
    \end{align*}
    \begin{align*}
        \nabla(\vec{a}\cdot \vec{r})&=\vec{a}\times(\nabla\times\vec{r})+(\vec{a}\cdot\nabla)\vec{r}+\vec{r}\times(\nabla\times\vec{a})+(\vec{r}\cdot\nabla)\vec{a}\\
        &=0+0+0+3\vec{a}\\
        &=3\vec{a}
    \end{align*}
    \begin{align*}
        \nabla\cdot[\vec{E}_0\sin (\vec{k}\cdot\vec{r})]&=\sin (\vec{k}\cdot\vec{r})(\nabla\cdot\vec{E}_0)+\vec{E}_0\cdot(\nabla\sin (\vec{k}\cdot\vec{r}))\\
        &=0+\vec{E}_0\cdot\vec{k}\cos(\vec{k}\cdot\vec{r})\\
        &=\vec{E}_0\cdot\vec{k}\cos(\vec{k}\cdot\vec{r})
    \end{align*}
    \begin{align*}
        \nabla\times[\vec{E}_0\sin (\vec{k}\cdot\vec{r})]&=(\nabla\sin (\vec{k}\cdot\vec{r}))\times\vec{E}_0+\sin (\vec{k}\cdot\vec{r})(\nabla\times\vec{E}_0)\\
        &=\vec{k}\cos(\vec{k}\cdot\vec{r})\times\vec{E}_0\\
        &=\cos(\vec{k}\cdot\vec{r})\vec{k}\times\vec{E}_0
    \end{align*}
\end{sol}\par

\begin{sol}[3]
    欲证$\int\limits_V\vec{A}\d V=0$,即证式
    \begin{align*}
        \vec{c}\cdot\int\limits_V\vec{A}\d V=0
    \end{align*}
    对于任意常矢量$\vec{c}$成立,我们构造一个新矢量场$\vec{F}$,定义为
    \begin{align*}
        \vec{F}=(\vec{c}\cdot \vec{r})\vec{A}
    \end{align*}
    则
    \begin{align*}
        \nabla\cdot\vec{F}=(\nabla\cdot\vec{A})(\vec{c}\cdot\vec{r})+\vec{A}\cdot\nabla(\vec{c}\cdot\vec{r})\\
    \end{align*}
    又因为$\nabla\cdot\vec{A}=0$,故
    \begin{align*}
        \nabla\cdot\vec{F}&=\vec{A}\cdot\nabla(\vec{c}\cdot\vec{r})\\
        &=\vec{A}\cdot\vec{c}
    \end{align*}
    故
    \begin{align*}
        \vec{c}\cdot\int\limits_V\vec{A}\d V&=\int\limits_V\vec{c}\cdot\vec{A}\d V\\
        &=\int\limits_V\nabla\cdot\vec{F}\d V\\
        &=\int\limits_S\vec{F}\cdot\d \vec{S}\\
        &=\int\limits_S(\vec{c}\cdot \vec{r})\vec{A}\cdot\d \vec{S}\\
    \end{align*}
    又因为$\vec{A}\cdot\d \vec{S}=0$,故$\vec{c}\cdot\int\limits_V\vec{A}\d V=0$,又因为此处$\vec{c}$是任取的,故原式得证。
\end{sol}\par

\begin{sol}[4]
    一维波:
    \begin{align*}
        \frac{\partial^2u}{\partial x^2}-\frac{1}{c^2}\frac{\partial^2u}{\partial t^2}=0
    \end{align*}
    令$v=\frac{\partial u}{\partial t},w=c\frac{\partial u}{\partial x}$,则有
    \begin{align*}
        \frac{\partial v}{\partial t}&=c\frac{\partial w}{\partial x}\\
        c\frac{\partial v}{\partial x}&=\frac{\partial w}{\partial t}
    \end{align*}
    一维麦克斯韦方程可写为
    \begin{align*}
        \frac{\partial E}{\partial x}&=-\frac{\partial B}{\partial t}\\
        \frac{\partial B}{\partial x}&=-\varepsilon_0\mu_0\frac{\partial E}{\partial t}\\
    \end{align*}
    异:$v,w$代表$u$的时空变化率,而$B,E$表示电磁场强度。\\
    同:数学结构相同,均允许波动解。
\end{sol}\par

\begin{sol}[5]
   \begin{align*}
    |A-\lambda I|&=\lambda ^4+ \left(-a^2-b^2-c^2+d^2+e^2+f^2\right)\lambda ^2-a^2 f^2-2 a b e f-2 a c d f-b^2 e^2-2 b c d e-c^2 d^2
   \end{align*}
   \begin{align*}
    |B-\lambda I|&=\lambda ^4+\left(a^2+b^2+c^2+d^2+e^2+f^2\right)\lambda ^2 +a^2 f^2+2 a b e f+2 a c d f+b^2 e^2+2 b c d e+c^2 d^2
   \end{align*}
\end{sol}\par

\begin{sol}[6]
    (a)球坐标:设$\rho=c\delta(r-a)$,则有
    \begin{align*}
        \int_{0}^{2\pi}\d\varphi\int_{0}^{\pi}\sin\theta\d\theta\int_{0}^{R}r^2c\delta(r-a)\d r&=q\\
        4\pi a^2c&=q\\
        c&=\frac{q}{4\pi a^2}
    \end{align*}
    即$\rho=\frac{q}{4\pi a^2}\delta(r-a)$\\
    (b)柱坐标:设$\rho=c\delta(r-a)(0<z<\lambda)$,则有
    \begin{align*}
        \int_{0}^{\lambda}\d z \int_{0}^{2\pi}\d\theta\int_{0}^{R}cr\delta(r-a)\d r&=q\\
        2\pi \lambda ac&=q\\
        c&=\frac{q}{ 2\pi \lambda a}
    \end{align*}
    即$\rho=\frac{q}{2\pi \lambda a}\delta(r-a)(0<z<\lambda)$\\
    (c)柱坐标:设$\rho=c\delta(z)(0<r<a)$,则有
    \begin{align*}
        \int_{-\lambda}^{\lambda}c\delta(z)\d z \int_{0}^{2\pi}\d\theta\int_{0}^{a}r\d r&=q\\
        \pi a^2c&=q\\
        c&=\frac{q}{\pi a^2}
    \end{align*}
    即$\rho=\frac{q}{\pi a^2}\delta(z)(0<r<a)$
 \end{sol}\par

 \begin{sol}[7]
    
    \begin{align*}
    I&=\frac{2}{|4-5|}+\frac{3}{|6-5|}\\
    &=5
    \end{align*}
    
 \end{sol}\par
 
\begin{sol}[8]
    \begin{align*}
    T_{ik}a_ib_k-T_{ik}a_kb_i&=T_{ik}a_ib_k-T_{ki}a_ib_k\\
    &=(T_{ik}-T_{ki})a_ib_k
    \end{align*}
    \begin{align*}
        2\vec{\omega}\cdot(\vec{a}\times\vec{b})&=2\varepsilon_{ijk}\omega_ia_jb_k\\
        &=2\varepsilon_{jik}\omega_ja_ib_k
    \end{align*}
    欲使等式成立即使
    \begin{align*}
        2\varepsilon_{jik}\omega_j&=T_{ik}-T_{ki}\\
        \varepsilon_{jik}\omega_j&=\frac{T_{ik}-T_{ki}}{2}
    \end{align*}
    两边均为反对称矩阵故只需为$\omega$选取合适的分量,即可使得等式成立。
\end{sol}\par

\begin{sol}[9]
    设三条带电直线交点为$(0,0),(a,0),(b,c)$,选取这三点组成的三角形角平分线的交点为零电势点,则平面上除直线上的任意一点电势可写为
    \begin{align*}
        \varphi=k\ln \frac{d_1d_2d_3}{r^3}
    \end{align*}
    式子中的$k$为一与电荷线密度相关的常数,$r$为零电势点距这三条直线的距离且为常数。为了方便,我们不妨只考察$\ln d_1d_2d_3$。
    即\begin{align*}
        \varphi\cong \ln d_1d_2d_3\cong \ln d_1^2d_2^2d_3^2=\ln \left(y^2 \left(y-\frac{c x}{b}\right)^2 \left(\frac{y (b-a)}{c}+a-x\right)^2\right)
    \end{align*}
    则场强为
    \begin{align*}
        \vec{E}&=-\nabla\varphi\\
        &=-\frac{2 c (a (y-c)-2 b y+2 c x)}{(c x-b y) (a (y-c)-b y+c x)}\vec{e}_x-\frac{a \left(4 c y (b+x)-6 b y^2-2 c^2 x\right)+6 b^2 y^2-8 b c x y+2 c^2 x^2}{y (b y-c x) (a (c-y)+b y-c x)}\vec{e}_y\\
    \end{align*}
    令$\vec{E}=0$,则解得
    \begin{align*}
        x=\frac{a+b}{3},y=\frac{c}{3}
    \end{align*}
    即该点位于三线交点所组成的三角形的重心。
\end{sol}\par
\end{document}