\documentclass{phyasgn}
\phyasgn{
  stuname = 姚昊廷,           % 设置学生姓名
  stunum = 22322091,      % 设置学号
  setasgnnum = 3,           % 设置课程次数
  classname = 电动力学,     % 设置课程名称
}

\usepackage{listings}
\usepackage{tikz}
\usepackage{amssymb}
\usepackage{t-angles}
\usepackage{amssymb}
\usepackage{tikz}
\usepackage{mathrsfs}
\usepackage{pifont}
\usepackage{subfigure}
\usepackage{caption}
%\usepackage{autobreak} 
%\usepackage{fixdif} 
\usetikzlibrary{quotes,angles}
\usetikzlibrary{calc}
\usetikzlibrary{decorations.pathreplacing}
\lstset{numbers=left,basicstyle=\ttfamily,columns=flexible}
\makeatletter
\newcommand{\rmnum}[1]{\romannumeral #1}
\newcommand{\Rmnum}[1]{\expandafter\@slowromancap\romannumeral #1@}
\renewcommand{\i}{\mathrm{i}}
\makeatother
\allowdisplaybreaks[4]%允许公式跨页


\begin{document}

\begin{sol}[1]
  \begin{align*}
    \frac{\d p}{\d t}&=\frac{\d }{\d t}\int_V\rho \vec{x}'\d V'\\
    &=\int_V\frac{\d \rho \vec{x}'}{\d t}\d V'\\
    &=\int_V(\frac{\partial \rho \vec{x}'}{\partial t}+\frac{\partial \rho \vec{x}'}{\partial \vec{x}'}\frac{\d \vec{x}'}{\d t})\d V'\\
    &=\int_V\frac{\partial \rho}{\partial t}\vec{x}'\d V'\\
    &=-\int_V\nabla'\cdot\vec{j}\vec{x}'\d V'
  \end{align*}  
  又因为
  \begin{align*}
    \nabla'\cdot (\vec{x}'\vec{j})&=\partial_ix'_jj_i\\
    &=x'_j\partial_ij_i+j_i\partial_ix'_j\\
    &=x'_j\partial_ij_i+j_i\delta_{ij}\\
    &=x'_j\partial_ij_i+j_j\\
    &=(\nabla'\cdot \vec{j})\vec{x'}+\vec{j}
  \end{align*}
  故\begin{align*}
    -\int_V\nabla'\cdot\vec{j}\vec{x}'\d V'&=\int_V\vec{j}\d V'-\int_V \nabla'\cdot (\vec{x}'\vec{j})\d V'\\
    &=\int_V\vec{j}\d V'-\int_{\partial V} \vec{x}'\vec{j}\cdot\d \vec{S}
  \end{align*}
  又体系电荷守恒,故在$\partial V$上$\vec{j}\cdot\d \vec{S}=0$,故
  \begin{align*}
    \nabla'\cdot (\vec{x}'\vec{j})=\int_V\vec{j}\d V'
  \end{align*}
\end{sol}\par

\begin{sol}[2]
  \begin{align*}
    \nabla\times\vec{A}&=\nabla(\frac{1}{r^3})\times(\vec{m}\times\vec{r})+\frac{1}{r^3}\nabla\times(\vec{m}\times\vec{r})\\
    &=\frac{-3\vec{r}}{r^5}\times(\vec{m}\times\vec{r})+\frac{1}{r^3}[(\vec{r}\cdot\nabla)\vec{m}+(\nabla\cdot\vec{r})\vec{m}-(\vec{m}\cdot\nabla)\vec{r}-(\nabla\cdot\vec{m})\vec{r}]\\
    &=\frac{-3[r^2\vec{m}-(\vec{r}\cdot\vec{m})\vec{r}]}{r^5}+\frac{2\vec{m}}{r^3}\\
    &=\frac{-3r^2\vec{m}}{r^3}+\frac{3(\vec{r}\cdot\vec{m})\vec{r}}{r^5}+\frac{2\vec{m}}{r^3}\\
    &=\frac{3(\vec{r}\cdot\vec{m})\vec{r}}{r^5}-\frac{\vec{m}}{r^3}
  \end{align*}
  \begin{align*}
    \nabla\varphi&=\frac{1}{r^3}\nabla(\vec{m}\cdot\vec{r})+(\vec{m}\cdot\vec{r})\nabla\frac{1}{r^3}\\
    &=\frac{1}{r^3}[\vec{m}\times(\nabla\times\vec{r})+(\vec{m}\cdot\nabla)\vec{r}+\vec{r}\times(\nabla\times\vec{m})+(\vec{r}\cdot\nabla)\vec{m}]-\frac{3(\vec{m}\cdot\vec{r})\vec{r}}{r^5}\\
    &=\frac{\vec{m}}{r^3}-\frac{3(\vec{r}\cdot\vec{m})\vec{r}}{r^5}
  \end{align*}  
  故
  \begin{align*}
    \nabla\times\vec{A}=-\nabla\varphi
  \end{align*}
\end{sol}\par

\begin{sol}[4]
  $$\begin{aligned}
        \vec{F}_{12}&=\frac{\mu_0}{4\pi}\oint\limits_{(L_1)}\oint\limits_{(L_2)}\frac{I_1I_2\d \vec{l}_1\times (\d \vec{l}_2\times\hat{\mathbf{r}}_{12})}{r_{12}^2}\\
        &=\frac{\mu_0}{4\pi}\oint\limits_{(L_1)}\oint\limits_{(L_2)}\frac{I_1I_2[(\d \vec{l}_1\cdot \hat{\mathbf{r}}_{12})\d \vec{l}_2-(\d \vec{l}_1\cdot \d \vec{l}_2)\hat{\mathbf{r}}_{12}]}{r_{12}^2}\\
        &=-\frac{\mu_0}{4\pi}\oint\limits_{(L_1)}\oint\limits_{(L_2)}\frac{I_1I_2(\d \vec{l}_1\cdot \d \vec{l}_2)\hat{\mathbf{r}}_{12}}{r_{12}^2}
    \end{aligned}$$
    又因为被积函数连续,故积分可交换顺序,即
    $$\vec{F}_{12}=-\frac{\mu_0}{4\pi}\oint\limits_{(L_2)}\oint\limits_{(L_1)}\frac{I_1I_2(\d \vec{l}_1\cdot \d \vec{l}_2)\hat{\mathbf{r}}_{12}}{r_{12}^2}$$
    同理
    $$\vec{F}_{21}=-\frac{\mu_0}{4\pi}\oint\limits_{(L_2)}\oint\limits_{(L_1)}\frac{I_1I_2(\d \vec{l}_1\cdot \d \vec{l}_2)\hat{\mathbf{r}}_{21}}{r_{21}^2}$$
    又因为$\hat{\mathbf{r}}_{21}=-\hat{\mathbf{r}}_{12}$,故
    $$\vec{F}_{12}=-\vec{F}_{21}$$
\end{sol}\par

\begin{sol}[3]
  设抛物线方程为$y=ax^2(z=0)$,则其焦点为$(0,\frac{1}{4a},0)$。在其上一点$(x,ax^2,0)$的电流元为$I\d \vec{l}=(I\d x,I2ax\d x,0)$。
  故其焦点处的磁感应强度为
  \begin{align*}
    \vec{B}&=\int\frac{\mu_0}{4\pi}\frac{I\d \vec{l}\times\vec{r}}{r^3}\\
    &=-\int_{-\infty}^{\infty}\frac{\mu_0I}{4\pi}\frac{ax^2+\frac{1}{4a}}{\sqrt{x^2+(ax^2-\frac{1}{4a})^2}^3}\d x\vec{e}_z\\
    &=-\frac{\mu_0I}{4\pi}4a\pi\vec{e}_z\\
    &=-a\mu_0I\vec{e}_z
  \end{align*}
\end{sol}\par
\end{document}