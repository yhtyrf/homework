\documentclass{phyasgn}
\phyasgn{
  stuname = 姚昊廷,           % 设置学生姓名
  stunum = 22322091,      % 设置学号
  setasgnnum = 3,           % 设置课程次数
  classname = 电动力学,     % 设置课程名称
}

\usepackage{listings}
\usepackage{tikz}
\usepackage{amssymb}
\usepackage{t-angles}
\usepackage{amssymb}
\usepackage{tikz}
\usepackage{mathrsfs}
\usepackage{pifont}
\usepackage{subfigure}
\usepackage{caption}
%\usepackage{autobreak} 
%\usepackage{fixdif} 
\usetikzlibrary{quotes,angles}
\usetikzlibrary{calc}
\usetikzlibrary{decorations.pathreplacing}
\lstset{numbers=left,basicstyle=\ttfamily,columns=flexible}
\makeatletter
\newcommand{\rmnum}[1]{\romannumeral #1}
\newcommand{\Rmnum}[1]{\expandafter\@slowromancap\romannumeral #1@}
\renewcommand{\i}{\mathrm{i}}
\makeatother
\allowdisplaybreaks[4]%允许公式跨页


\begin{document}

\begin{sol}[1]
  \begin{align*}
    \frac{\d p}{\d t}&=\frac{\d }{\d t}\int_V\rho \vec{x}'\d V'\\
    &=\int_V\frac{\d \rho \vec{x}'}{\d t}\d V'\\
    &=\int_V(\frac{\partial \rho \vec{x}'}{\partial t}+\frac{\partial \rho \vec{x}'}{\partial \vec{x}'}\frac{\d \vec{x}'}{\d t})\d V'\\
    &=\int_V\frac{\partial \rho}{\partial t}\vec{x}'\d V'\\
    &=-\int_V\nabla'\cdot\vec{j}\vec{x}'\d V'
  \end{align*}  
  又因为
  \begin{align*}
    \nabla'\cdot (\vec{j}\vec{x}')&=\partial_ij_ix'_j\\
    &=x'_j\partial_ij_i+j_i\partial_ix'_j\\
    &=x'_j\partial_ij_i+j_i\delta_{ij}\\
    &=x'_j\partial_ij_i+j_j\\
    &=(\nabla'\cdot \vec{j})\vec{x'}+\vec{j}
  \end{align*}
  故\begin{align*}
    -\int_V\nabla'\cdot\vec{j}\vec{x}'\d V'&=\int_V\vec{j}\d V'-\int_V \nabla'\cdot (\vec{j}\vec{x}')\d V'\\
    &=\int_V\vec{j}\d V'-\int_{\partial V} \d \vec{S}\cdot\vec{j}\vec{x}'
  \end{align*}
  又体系电荷守恒,故在$\partial V$上$\vec{j}\cdot\d \vec{S}=0$,故
  \begin{align*}
    \nabla'\cdot (\vec{x}'\vec{j})=\int_V\vec{j}\d V'
  \end{align*}
\end{sol}\par

\begin{sol}[2]
  \begin{align*}
    \nabla\times\vec{A}&=\nabla(\frac{1}{r^3})\times(\vec{m}\times\vec{r})+\frac{1}{r^3}\nabla\times(\vec{m}\times\vec{r})\\
    &=\frac{-3\vec{r}}{r^5}\times(\vec{m}\times\vec{r})+\frac{1}{r^3}[(\vec{r}\cdot\nabla)\vec{m}+(\nabla\cdot\vec{r})\vec{m}-(\vec{m}\cdot\nabla)\vec{r}-(\nabla\cdot\vec{m})\vec{r}]\\
    &=\frac{-3[r^2\vec{m}-(\vec{r}\cdot\vec{m})\vec{r}]}{r^5}+\frac{2\vec{m}}{r^3}\\
    &=\frac{-3\vec{m}}{r^3}+\frac{3(\vec{r}\cdot\vec{m})\vec{r}}{r^5}+\frac{2\vec{m}}{r^3}\\
    &=\frac{3(\vec{r}\cdot\vec{m})\vec{r}}{r^5}-\frac{\vec{m}}{r^3}
  \end{align*}
  \begin{align*}
    \nabla\varphi&=\frac{1}{r^3}\nabla(\vec{m}\cdot\vec{r})+(\vec{m}\cdot\vec{r})\nabla\frac{1}{r^3}\\
    &=\frac{1}{r^3}[\vec{m}\times(\nabla\times\vec{r})+(\vec{m}\cdot\nabla)\vec{r}+\vec{r}\times(\nabla\times\vec{m})+(\vec{r}\cdot\nabla)\vec{m}]-\frac{3(\vec{m}\cdot\vec{r})\vec{r}}{r^5}\\
    &=\frac{\vec{m}}{r^3}-\frac{3(\vec{r}\cdot\vec{m})\vec{r}}{r^5}
  \end{align*}  
  故
  \begin{align*}
    \nabla\times\vec{A}=-\nabla\varphi
  \end{align*}
\end{sol}\par

\begin{sol}[4]
  $$\begin{aligned}
        \vec{F}_{12}&=\frac{\mu_0}{4\pi}\oint\limits_{(L_1)}\oint\limits_{(L_2)}\frac{I_1I_2\d \vec{l}_1\times (\d \vec{l}_2\times\hat{\mathbf{r}}_{12})}{r_{12}^2}\\
        &=\frac{\mu_0}{4\pi}\oint\limits_{(L_1)}\oint\limits_{(L_2)}\frac{I_1I_2[(\d \vec{l}_1\cdot \hat{\mathbf{r}}_{12})\d \vec{l}_2-(\d \vec{l}_1\cdot \d \vec{l}_2)\hat{\mathbf{r}}_{12}]}{r_{12}^2}\\
        &=-\frac{\mu_0}{4\pi}\oint\limits_{(L_1)}\oint\limits_{(L_2)}\frac{I_1I_2(\d \vec{l}_1\cdot \d \vec{l}_2)\hat{\mathbf{r}}_{12}}{r_{12}^2}
    \end{aligned}$$
    又因为被积函数连续,故积分可交换顺序,即
    $$\vec{F}_{12}=-\frac{\mu_0}{4\pi}\oint\limits_{(L_2)}\oint\limits_{(L_1)}\frac{I_1I_2(\d \vec{l}_1\cdot \d \vec{l}_2)\hat{\mathbf{r}}_{12}}{r_{12}^2}$$
    同理
    $$\vec{F}_{21}=-\frac{\mu_0}{4\pi}\oint\limits_{(L_2)}\oint\limits_{(L_1)}\frac{I_1I_2(\d \vec{l}_1\cdot \d \vec{l}_2)\hat{\mathbf{r}}_{21}}{r_{21}^2}$$
    又因为$\hat{\mathbf{r}}_{21}=-\hat{\mathbf{r}}_{12}$,故
    $$\vec{F}_{12}=-\vec{F}_{21}$$
\end{sol}\par

\begin{sol}[3]
  设抛物线方程为$y=ax^2(z=0)$,则其焦点为$(0,\frac{1}{4a},0)$。在其上一点$(x,ax^2,0)$的电流元为$I\d \vec{l}=(I\d x,I2ax\d x,0)$。
  故其焦点处的磁感应强度为
  \begin{align*}
    \vec{B}&=\int\frac{\mu_0}{4\pi}\frac{I\d \vec{l}\times\vec{r}}{r^3}\\
    &=-\int_{-\infty}^{\infty}\frac{\mu_0I}{4\pi}\frac{ax^2+\frac{1}{4a}}{\sqrt{x^2+(ax^2-\frac{1}{4a})^2}^3}\d x\vec{e}_z\\
    &=-\frac{\mu_0I}{4\pi}4a\pi\vec{e}_z\\
    &=-a\mu_0I\vec{e}_z
  \end{align*}
\end{sol}\par

\begin{sol}[4]
   因为所有场量均只与$z,t$相关,故麦克斯韦方程组可简化为
   \begin{align*}
    \frac{\partial E_z}{\partial z}&=0\\
    \frac{\partial E_y}{\partial z}\vec{e}_x-\frac{\partial E_x}{\partial z}\vec{e}_y&=\frac{\partial \vec{B}}{\partial t}\\
    -\frac{\partial B_y}{\partial z}\vec{e}_x+\frac{\partial B_x}{\partial z}\vec{e}_y&=\mu_0\varepsilon_0\frac{\partial\vec{E}}{\partial t}\\
    \frac{\partial B_z}{\partial z}&=0
   \end{align*}
   因为$\frac{\partial E_z}{\partial z}=0,\frac{\partial B_z}{\partial z}=0$故$E_z$为常数,同理$B_z$为常数,不妨将$E_z,B_z$均取为0。
   故可得到两组独立方程
   \begin{align*}
    \left\{\begin{matrix}
      \frac{\partial E_y}{\partial z}=\frac{\partial B_x}{\partial t}\\
      \frac{\partial B_x}{\partial z}=\mu_0\varepsilon_0\frac{\partial E_y}{\partial t}
    \end{matrix}\right.\\
    \left\{\begin{matrix}
      -\frac{\partial E_x}{\partial z}=\frac{\partial B_y}{\partial t}\\
      -\frac{\partial B_y}{\partial z}=\mu_0\varepsilon_0\frac{\partial E_x}{\partial t}
    \end{matrix}\right.
   \end{align*}
   消去第一组方程中的$B_x$可得到
   \begin{align*}
    \mu_0\varepsilon_0\frac{\partial^2 E_y}{\partial t^2}=\frac{\partial^2 E_y}{\partial z^2}
   \end{align*}
   此是一波动方程,可解出$E_y$,再代入原式,可解出$B_x$。此是一组解。同理第二组方程可解出第二组解$(E_x,B_y)$。
   故整个方程组的解由这两组独立的解组成。
\end{sol}\par

\begin{sol}[4]
  \begin{align*}
    \nabla\cdot\vec{E}'&=\cos\theta\nabla\cdot\vec{E}+c\sin\theta\nabla\cdot\vec{B}\\
    &=0\\
    \nabla\times\vec{E}'&=\cos\theta\nabla\times\vec{E}+c\sin\theta\nabla\times\vec{B}\\
    &=-\cos\theta\frac{\partial \vec{B}}{\partial t}+c\mu_0\varepsilon_0\sin\theta\frac{\partial\vec{E}}{\partial t}\\
    &=-\frac{\partial}{\partial t}(\cos\theta\vec{B}-c\mu_0\varepsilon_0\sin\theta\vec{E})\\
    &=-\frac{\partial}{\partial t}(\cos\theta\vec{B}-\sqrt{\mu_0\varepsilon_0}\sin\theta\vec{E})\\
    &=-\frac{\partial}{\partial t}(\cos\theta\vec{B}-\frac{1}{c}\sin\theta\vec{E})\\
    &=-\frac{\partial \vec{B}'}{\partial t}\\
    \nabla\cdot\vec{B}'&=\frac{-\sin\theta}{c}\nabla\cdot\vec{E}+\cos\theta\nabla\cdot\vec{B}\\
    &=0\\
    \nabla\times\vec{B}'&=\frac{-\sin\theta}{c}\nabla\times\vec{E}+\cos\theta\nabla\times\vec{B}\\
    &=\frac{\sin\theta}{c}\frac{\partial\vec{B}}{\partial t}+\cos\theta\mu_0\varepsilon_0\frac{\partial \vec{E}}{\partial t}\\
    &=\mu_0\varepsilon_0\frac{\partial}{\partial t}(\frac{\sin\theta}{\mu_0\varepsilon_0c}\vec{B}+\cos\theta\vec{E})\\
    &=\mu_0\varepsilon_0\frac{\partial}{\partial t}(\frac{\sin\theta}{\sqrt{\mu_0\varepsilon_0}}\vec{B}+\cos\theta\vec{E})\\
    &=\mu_0\varepsilon_0\frac{\partial}{\partial t}(c\sin\theta\vec{B}+\cos\theta\vec{E})\\
    &=\mu_0\varepsilon_0\frac{\partial \vec{E}'}{\partial t}
  \end{align*}
\end{sol}\par

\begin{sol}[4]
  \begin{align*}
    \vec{B}&=\nabla\times\vec{A}\\
    &=(\nabla\frac{1}{r(r-\vec{r}\cdot\vec{n})})\times(\vec{r}\times\vec{n})+\frac{1}{r(r-\vec{r}\cdot\vec{n})}\nabla\times(\vec{r}\times\vec{n})\\
    &=(\nabla\frac{1}{r^2}\frac{1}{1-\cos\theta})\times(\vec{r}\times\vec{n})-\frac{2\vec{n}}{r(r-\vec{r}\cdot\vec{n})}
  \end{align*}
  \begin{align*}
    \nabla\frac{1}{r^2}\frac{1}{1-\cos\theta}&=\frac{1}{1-\cos\theta}\nabla\frac{1}{r^2}+\frac{1}{r^2}\nabla\frac{1}{1-\cos\theta}\\
    &=\frac{1}{1-\cos\theta}\frac{-2\vec{r}}{r^4}+\frac{1}{r^2}\frac{1}{(1-\cos\theta)^2}\nabla\cos\theta\\
    &=\frac{-2\vec{r}}{r^3(r-\vec{r}\cdot\vec{n})}+\frac{1}{(r-\vec{r}\cdot\vec{n})^2}\nabla\frac{\vec{r}\cdot\vec{n}}{r}
  \end{align*}
  \begin{align*}
    \nabla\frac{\vec{r}\cdot\vec{n}}{r}&=\frac{1}{r}\nabla(\vec{r}\cdot\vec{n})+(\vec{r}\cdot\vec{n})\nabla\frac{1}{r}\\
    &=\frac{1}{r}[\vec{n}\times(\nabla\times\vec{r})+(\vec{n}\cdot\nabla)\vec{r}+\vec{r}\times(\nabla\times\vec{n})+(\vec{r}\cdot\nabla)\vec{n}]-(\vec{r}\cdot\vec{n})\frac{\vec{r}}{r^3}\\
    &=\frac{\vec{n}}{r}-\frac{(\vec{r}\cdot\vec{n})\vec{r}}{r^3}
  \end{align*}
  故
  \begin{align*}
    \vec{B}&=(\nabla\frac{1}{r^2}\frac{1}{1-\cos\theta})\times(\vec{r}\times\vec{n})-\frac{2\vec{n}}{r(r-\vec{r}\cdot\vec{n})}\\
    &=(\frac{-2\vec{r}}{r^3(r-\vec{r}\cdot\vec{n})}+\frac{1}{(r-\vec{r}\cdot\vec{n})^2}\nabla\frac{\vec{r}\cdot\vec{n}}{r})\times(\vec{r}\times\vec{n})-\frac{2\vec{n}}{r(r-\vec{r}\cdot\vec{n})}\\
    &=(\frac{-2\vec{r}}{r^3(r-\vec{r}\cdot\vec{n})}+\frac{1}{(r-\vec{r}\cdot\vec{n})^2}(\frac{\vec{n}}{r}-\frac{(\vec{r}\cdot\vec{n})\vec{r}}{r^3}))\times(\vec{r}\times\vec{n})-\frac{2\vec{n}}{r(r-\vec{r}\cdot\vec{n})}\\
    &=\frac{(\vec{r}\cdot\vec{n}-2r)\vec{r}+r^2\vec{n}}{r^3(r-\vec{r}\cdot\vec{n})^2}\times(\vec{r}\times\vec{n})-\frac{2\vec{n}}{r(r-\vec{r}\cdot\vec{n})}\\
    &=\frac{\vec{r}\cdot\vec{n}-2r}{r^3(r-\vec{r}\cdot\vec{n})^2}[(\vec{r}\cdot\vec{n})\vec{r}-r^2\vec{n}]+\frac{r^2}{r^3(r-\vec{r}\cdot\vec{n})^2}[\vec{r}-(\vec{n}\cdot\vec{r})\vec{n}]-\frac{2\vec{n}}{r(r-\vec{r}\cdot\vec{n})}\\
    &=\frac{[(\vec{r}\cdot\vec{n})^2-2(\vec{r}\cdot\vec{n})r+r^2]\vec{r}}{r^3(r-\vec{r}\cdot\vec{n})^2}+\frac{\vec{n}}{r(r-\vec{r}\cdot\vec{n})}
  \end{align*}
\end{sol}\par
\end{document}