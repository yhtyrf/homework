\documentclass{phyasgn}
\phyasgn{
  stuname = 姚昊廷,           % 设置学生姓名
  stunum = 22322091,      % 设置学号
  setasgnnum = 1,           % 设置课程次数
  classname = 电动力学,     % 设置课程名称
}

\usepackage{listings}
\usepackage{tikz}
\usepackage{amssymb}
\usepackage{t-angles}
\usepackage{amssymb}
\usepackage{tikz}
\usepackage{mathrsfs}
\usepackage{pifont}
\usepackage{subfigure}
\usepackage{caption}
%\usepackage{autobreak} 
%\usepackage{fixdif} 
\usetikzlibrary{quotes,angles}
\usetikzlibrary{calc}
\usetikzlibrary{decorations.pathreplacing}
\lstset{numbers=left,basicstyle=\ttfamily,columns=flexible}
\makeatletter
\newcommand{\rmnum}[1]{\romannumeral #1}
\newcommand{\Rmnum}[1]{\expandafter\@slowromancap\romannumeral #1@}
\renewcommand{\i}{\mathrm{i}}
\makeatother
\allowdisplaybreaks[4]%允许公式跨页


\begin{document}

\begin{sol}[1]
  电场强度可写为
    \begin{align*}
      E=\frac{q}{4\pi\varepsilon_0}\frac{1}{r^2}
    \end{align*}
  其法向导数为
  \begin{align*}
    \frac{\partial E}{\partial r}&=\frac{q}{4\pi\varepsilon_0}\frac{-2}{r^3}\\
    &=\frac{-2E}{r}
  \end{align*}
  故
  \begin{align*}
    \frac{1}{E}\frac{\partial E}{\partial n}=-\frac{2}{R}
  \end{align*}
\end{sol}\par

\begin{sol}[2]
  设该电偶极子在直角坐标下的电偶极矩为$\vec{p}=(ql,0,0)$,则产生的电势为
  \begin{align*}
    \varphi&=\frac{\vec{p}\cdot \vec{r}}{4\pi\varepsilon_0 r^3}\\
    &=\frac{qlx}{4\pi\varepsilon_0(x^2+y^2+z^2)^{\frac{3}{2}}}
  \end{align*}
  故电场强度为
  \begin{align*}
    \vec{E}&=-\nabla\varphi\\
    &=[\frac{3 l q x^2}{4\pi\varepsilon_0\left(x^2+y^2+z^2\right)^{5/2}}-\frac{l q}{4\pi\varepsilon_0\left(x^2+y^2+z^2\right)^{3/2}}]\vec{e}_x+\\
    &\frac{3 l q x y}{\left(x^2+y^2+z^2\right)^{5/2}}\vec{e}_y+(3 l q x z)/(x^2 + y^2 + z^2)^(5/2)\vec{e}_z
  \end{align*}
  故
  \begin{align*}
    \frac{\partial E_x}{\partial y}&=\frac{3 l q y}{4\pi\varepsilon_0\left(x^2+y^2+z^2\right)^{5/2}}-\frac{15 l q x^2 y}{4\pi\varepsilon_0\left(x^2+y^2+z^2\right)^{7/2}}\\
    \frac{\partial E_x}{\partial z}&=\frac{3 l q z}{4\pi\varepsilon_0\left(x^2+y^2+z^2\right)^{5/2}}-\frac{15 l q x^2 z}{4\pi\varepsilon_0\left(x^2+y^2+z^2\right)^{7/2}}\\
    \frac{\partial E_y}{\partial x}&=\frac{3 l q y}{4\pi\varepsilon_0\left(x^2+y^2+z^2\right)^{5/2}}-\frac{15 l q x^2 y}{4\pi\varepsilon_0\left(x^2+y^2+z^2\right)^{7/2}}\\
    \frac{\partial E_y}{\partial z}&=-\frac{15 l q x y z}{4\pi\varepsilon_0\left(x^2+y^2+z^2\right)^{7/2}}\\
    \frac{\partial E_z}{\partial x}&=\frac{3 l q y}{4\pi\varepsilon_0\left(x^2+y^2+z^2\right)^{5/2}}-\frac{15 l q x^2 y}{4\pi\varepsilon_0\left(x^2+y^2+z^2\right)^{7/2}}\\
    \frac{\partial E_z}{\partial y}&=-\frac{15 l q x y z}{4\pi\varepsilon_0\left(x^2+y^2+z^2\right)^{7/2}}\\
  \end{align*}
  因为
  \begin{align*}
    \frac{\partial E_x}{\partial y}&=\frac{\partial E_y}{\partial x}\\
    \frac{\partial E_x}{\partial z}&=\frac{\partial E_z}{\partial x}\\
    \frac{\partial E_y}{\partial z}&=\frac{\partial E_z}{\partial y}
  \end{align*}
  故
  \begin{align*}
    \nabla\times\vec{E}=0
  \end{align*}
\end{sol}\par

\begin{sol}[3]
  设该电荷位于坐标原点,则其产生的电势为
  \begin{align*}
    \varphi&=\frac{q}{4\pi\varepsilon_0 r}\\
    &=\frac{q}{4\pi\varepsilon_0}\frac{1}{\sqrt{x^2+y^2+z^2}}
  \end{align*}
  考察其穿过平面$z=a(-a<x<a,-a<y<a)$的电通量,只需计算电场的$z$方向分量
  \begin{align*}
    E_z&=\frac{\partial \varphi}{\partial z}\\
    &=-\frac{q}{4\pi\varepsilon_0}\frac{z}{\left(x^2+y^2+z^2\right)^{3/2}}
  \end{align*}
  故其其穿过平面$z=a(-a<x<a,-a<y<a)$的电通量为
  \begin{align*}
    I&=\iint\frac{q}{4\pi\varepsilon_0}\frac{a}{\left(x^2+y^2+a^2\right)^{3/2}}\d x\d y\\
    &=\frac{aq}{4\pi\varepsilon_0}\int_{-a}^{a}\d y\int_{-a}^{a}\frac{1}{\left(x^2+y^2+a^2\right)^{3/2}}\d x\\
    &=\frac{aq}{4\pi\varepsilon_0}\int_{-a}^{a}\frac{2 a}{\left(a^2+y^2\right) \sqrt{2 a^2+y^2}}\d y\\
    &=\frac{aq}{4\pi\varepsilon_0}\frac{2 \pi }{3 a}\\
    &=\frac{q}{6\varepsilon_0}
  \end{align*}
\end{sol}\par

\begin{sol}[4]
  由对称性知,该点电场方向必为径向,故只需计算球上各点在该点产生电场的径向分量,取圆环微元
  该微元带电量为
  \begin{align*}
    \d q=2\pi R^2\sigma\sin\theta\d\theta
  \end{align*}
  其在考察点产生的电场强度为
  \begin{align*}
    \d E&=\frac{\d q}{4\pi\varepsilon_0}\frac{R-R\cos\theta}{[R^2\sin^2\theta+(R-R\cos\theta)^2]^{\frac{3}{2}}}\\
    &=\frac{2\pi R^2\sigma\sin\theta\d\theta}{4\pi\varepsilon_0}\frac{R-R\cos\theta}{[R^2\sin^2\theta+(R-R\cos\theta)^2]^{\frac{3}{2}}}
  \end{align*}
  故
  \begin{align*}
    E&=\int \d E\\
    &=\int_{\theta_0}^{\pi}\frac{2\pi R^2\sigma\sin\theta\d\theta}{4\pi\varepsilon_0}\frac{R-R\cos\theta}{[R^2\sin^2\theta+(R-R\cos\theta)^2]^{\frac{3}{2}}}\\
    &=\frac{\sigma}{2\varepsilon_0}\int_{\theta_0}^{\pi}\frac{\sin\theta(1-\cos\theta)\d\theta}{[\sin^2\theta+(1-\cos\theta)^2]^{\frac{3}{2}}}\\
    &=\frac{\sigma}{2\varepsilon_0}(1-\sin\frac{\theta_0}{2})
  \end{align*}
  故该点电场强度为
  \begin{align*}
    E=\frac{\sigma}{2\varepsilon_0}(1-\sin\frac{\theta_0}{2})
  \end{align*}
  方向沿径向
\end{sol}\par
\end{document}